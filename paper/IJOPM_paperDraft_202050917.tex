% Options for packages loaded elsewhere
% Options for packages loaded elsewhere
\PassOptionsToPackage{unicode}{hyperref}
\PassOptionsToPackage{hyphens}{url}
\PassOptionsToPackage{dvipsnames,svgnames,x11names}{xcolor}
%
\documentclass[
  12pt,
  letterpaper,
  DIV=11,
  numbers=noendperiod]{scrartcl}
\usepackage{xcolor}
\usepackage[margin=1in]{geometry}
\usepackage{amsmath,amssymb}
\setcounter{secnumdepth}{5}
\usepackage{iftex}
\ifPDFTeX
  \usepackage[T1]{fontenc}
  \usepackage[utf8]{inputenc}
  \usepackage{textcomp} % provide euro and other symbols
\else % if luatex or xetex
  \usepackage{unicode-math} % this also loads fontspec
  \defaultfontfeatures{Scale=MatchLowercase}
  \defaultfontfeatures[\rmfamily]{Ligatures=TeX,Scale=1}
\fi
\usepackage{lmodern}
\ifPDFTeX\else
  % xetex/luatex font selection
\fi
% Use upquote if available, for straight quotes in verbatim environments
\IfFileExists{upquote.sty}{\usepackage{upquote}}{}
\IfFileExists{microtype.sty}{% use microtype if available
  \usepackage[]{microtype}
  \UseMicrotypeSet[protrusion]{basicmath} % disable protrusion for tt fonts
}{}
\makeatletter
\@ifundefined{KOMAClassName}{% if non-KOMA class
  \IfFileExists{parskip.sty}{%
    \usepackage{parskip}
  }{% else
    \setlength{\parindent}{0pt}
    \setlength{\parskip}{6pt plus 2pt minus 1pt}}
}{% if KOMA class
  \KOMAoptions{parskip=half}}
\makeatother
% Make \paragraph and \subparagraph free-standing
\makeatletter
\ifx\paragraph\undefined\else
  \let\oldparagraph\paragraph
  \renewcommand{\paragraph}{
    \@ifstar
      \xxxParagraphStar
      \xxxParagraphNoStar
  }
  \newcommand{\xxxParagraphStar}[1]{\oldparagraph*{#1}\mbox{}}
  \newcommand{\xxxParagraphNoStar}[1]{\oldparagraph{#1}\mbox{}}
\fi
\ifx\subparagraph\undefined\else
  \let\oldsubparagraph\subparagraph
  \renewcommand{\subparagraph}{
    \@ifstar
      \xxxSubParagraphStar
      \xxxSubParagraphNoStar
  }
  \newcommand{\xxxSubParagraphStar}[1]{\oldsubparagraph*{#1}\mbox{}}
  \newcommand{\xxxSubParagraphNoStar}[1]{\oldsubparagraph{#1}\mbox{}}
\fi
\makeatother

\usepackage{color}
\usepackage{fancyvrb}
\newcommand{\VerbBar}{|}
\newcommand{\VERB}{\Verb[commandchars=\\\{\}]}
\DefineVerbatimEnvironment{Highlighting}{Verbatim}{commandchars=\\\{\}}
% Add ',fontsize=\small' for more characters per line
\usepackage{framed}
\definecolor{shadecolor}{RGB}{241,243,245}
\newenvironment{Shaded}{\begin{snugshade}}{\end{snugshade}}
\newcommand{\AlertTok}[1]{\textcolor[rgb]{0.68,0.00,0.00}{#1}}
\newcommand{\AnnotationTok}[1]{\textcolor[rgb]{0.37,0.37,0.37}{#1}}
\newcommand{\AttributeTok}[1]{\textcolor[rgb]{0.40,0.45,0.13}{#1}}
\newcommand{\BaseNTok}[1]{\textcolor[rgb]{0.68,0.00,0.00}{#1}}
\newcommand{\BuiltInTok}[1]{\textcolor[rgb]{0.00,0.23,0.31}{#1}}
\newcommand{\CharTok}[1]{\textcolor[rgb]{0.13,0.47,0.30}{#1}}
\newcommand{\CommentTok}[1]{\textcolor[rgb]{0.37,0.37,0.37}{#1}}
\newcommand{\CommentVarTok}[1]{\textcolor[rgb]{0.37,0.37,0.37}{\textit{#1}}}
\newcommand{\ConstantTok}[1]{\textcolor[rgb]{0.56,0.35,0.01}{#1}}
\newcommand{\ControlFlowTok}[1]{\textcolor[rgb]{0.00,0.23,0.31}{\textbf{#1}}}
\newcommand{\DataTypeTok}[1]{\textcolor[rgb]{0.68,0.00,0.00}{#1}}
\newcommand{\DecValTok}[1]{\textcolor[rgb]{0.68,0.00,0.00}{#1}}
\newcommand{\DocumentationTok}[1]{\textcolor[rgb]{0.37,0.37,0.37}{\textit{#1}}}
\newcommand{\ErrorTok}[1]{\textcolor[rgb]{0.68,0.00,0.00}{#1}}
\newcommand{\ExtensionTok}[1]{\textcolor[rgb]{0.00,0.23,0.31}{#1}}
\newcommand{\FloatTok}[1]{\textcolor[rgb]{0.68,0.00,0.00}{#1}}
\newcommand{\FunctionTok}[1]{\textcolor[rgb]{0.28,0.35,0.67}{#1}}
\newcommand{\ImportTok}[1]{\textcolor[rgb]{0.00,0.46,0.62}{#1}}
\newcommand{\InformationTok}[1]{\textcolor[rgb]{0.37,0.37,0.37}{#1}}
\newcommand{\KeywordTok}[1]{\textcolor[rgb]{0.00,0.23,0.31}{\textbf{#1}}}
\newcommand{\NormalTok}[1]{\textcolor[rgb]{0.00,0.23,0.31}{#1}}
\newcommand{\OperatorTok}[1]{\textcolor[rgb]{0.37,0.37,0.37}{#1}}
\newcommand{\OtherTok}[1]{\textcolor[rgb]{0.00,0.23,0.31}{#1}}
\newcommand{\PreprocessorTok}[1]{\textcolor[rgb]{0.68,0.00,0.00}{#1}}
\newcommand{\RegionMarkerTok}[1]{\textcolor[rgb]{0.00,0.23,0.31}{#1}}
\newcommand{\SpecialCharTok}[1]{\textcolor[rgb]{0.37,0.37,0.37}{#1}}
\newcommand{\SpecialStringTok}[1]{\textcolor[rgb]{0.13,0.47,0.30}{#1}}
\newcommand{\StringTok}[1]{\textcolor[rgb]{0.13,0.47,0.30}{#1}}
\newcommand{\VariableTok}[1]{\textcolor[rgb]{0.07,0.07,0.07}{#1}}
\newcommand{\VerbatimStringTok}[1]{\textcolor[rgb]{0.13,0.47,0.30}{#1}}
\newcommand{\WarningTok}[1]{\textcolor[rgb]{0.37,0.37,0.37}{\textit{#1}}}

\usepackage{longtable,booktabs,array}
\usepackage{calc} % for calculating minipage widths
% Correct order of tables after \paragraph or \subparagraph
\usepackage{etoolbox}
\makeatletter
\patchcmd\longtable{\par}{\if@noskipsec\mbox{}\fi\par}{}{}
\makeatother
% Allow footnotes in longtable head/foot
\IfFileExists{footnotehyper.sty}{\usepackage{footnotehyper}}{\usepackage{footnote}}
\makesavenoteenv{longtable}
\usepackage{graphicx}
\makeatletter
\newsavebox\pandoc@box
\newcommand*\pandocbounded[1]{% scales image to fit in text height/width
  \sbox\pandoc@box{#1}%
  \Gscale@div\@tempa{\textheight}{\dimexpr\ht\pandoc@box+\dp\pandoc@box\relax}%
  \Gscale@div\@tempb{\linewidth}{\wd\pandoc@box}%
  \ifdim\@tempb\p@<\@tempa\p@\let\@tempa\@tempb\fi% select the smaller of both
  \ifdim\@tempa\p@<\p@\scalebox{\@tempa}{\usebox\pandoc@box}%
  \else\usebox{\pandoc@box}%
  \fi%
}
% Set default figure placement to htbp
\def\fps@figure{htbp}
\makeatother





\setlength{\emergencystretch}{3em} % prevent overfull lines

\providecommand{\tightlist}{%
  \setlength{\itemsep}{0pt}\setlength{\parskip}{0pt}}



 
\usepackage[backend=biber,natbib =
true,style=apa,sorting=nyt,maxcitenames=2]{biblatex}
\addbibresource{references.bib}
\addbibresource{MyLibrary2025-08-25.bib}


% Do NOT load biblatex here.
% (Optional) Language mapping recommended for APA:
\DeclareLanguageMapping{english}{english-apa}
\usepackage{amsthm}
\usepackage{amsmath}
\usepackage{amsfonts}
\usepackage{amssymb}
\usepackage{float}
\usepackage{caption}
\usepackage{subcaption}
\usepackage{stmaryrd}
% Numbering is tied to sections (e.g., Proposition 2.1, 2.2, etc.)
% Theorems and Propositions (italic)
\theoremstyle{plain}
\newtheorem{theorem}{Theorem}[section]
\newtheorem{proposition}[theorem]{Proposition}
% Definitions and Corollaries (upright)
\theoremstyle{definition}
\newtheorem{definition}{Definition}
\newtheorem{corollary}{Corollary}
\newtheorem{example}{Example}
% \newtheorem{algorithm}[theorem]{Algorithm}
% Optional: formatting for the proof environment
\renewenvironment{proof}
   {\par\noindent\textbf{Proof.}\ }
   {\hfill$\blacksquare$\par}


\usepackage[ruled,vlined,linesnumbered]{algorithm2e}
\KOMAoption{captions}{tableheading}
\makeatletter
\@ifpackageloaded{caption}{}{\usepackage{caption}}
\AtBeginDocument{%
\ifdefined\contentsname
  \renewcommand*\contentsname{Table of contents}
\else
  \newcommand\contentsname{Table of contents}
\fi
\ifdefined\listfigurename
  \renewcommand*\listfigurename{List of Figures}
\else
  \newcommand\listfigurename{List of Figures}
\fi
\ifdefined\listtablename
  \renewcommand*\listtablename{List of Tables}
\else
  \newcommand\listtablename{List of Tables}
\fi
\ifdefined\figurename
  \renewcommand*\figurename{Figure}
\else
  \newcommand\figurename{Figure}
\fi
\ifdefined\tablename
  \renewcommand*\tablename{Table}
\else
  \newcommand\tablename{Table}
\fi
}
\@ifpackageloaded{float}{}{\usepackage{float}}
\floatstyle{ruled}
\@ifundefined{c@chapter}{\newfloat{codelisting}{h}{lop}}{\newfloat{codelisting}{h}{lop}[chapter]}
\floatname{codelisting}{Listing}
\newcommand*\listoflistings{\listof{codelisting}{List of Listings}}
\makeatother
\makeatletter
\makeatother
\makeatletter
\@ifpackageloaded{caption}{}{\usepackage{caption}}
\@ifpackageloaded{subcaption}{}{\usepackage{subcaption}}
\makeatother
\usepackage{bookmark}
\IfFileExists{xurl.sty}{\usepackage{xurl}}{} % add URL line breaks if available
\urlstyle{same}
\hypersetup{
  pdftitle={Quantifying Hidden Exploitation: Dual-Method Prevalence Estimates of Modern Slavery Risk Among UK Domestic Workers},
  pdfauthor={Caroline Emberson; Scott Moser},
  colorlinks=true,
  linkcolor={black},
  filecolor={Maroon},
  citecolor={RoyalBlue},
  urlcolor={BrickRed},
  pdfcreator={LaTeX via pandoc}}


\title{Quantifying Hidden Exploitation: Dual-Method Prevalence Estimates
of Modern Slavery Risk Among UK Domestic Workers\footnote{Authors' names
  are listed in alphabetical order.}}
\author{Caroline Emberson \and Scott Moser}
\date{18, September 2025}
\begin{document}
\maketitle

\renewcommand*\contentsname{Table of contents}
{
\hypersetup{linkcolor=}
\setcounter{tocdepth}{3}
\tableofcontents
}

\newpage

Introduction

Labour exploitation has been defined as `work situations that deviate 
significantly from standard working conditions as defined by legislation
or other binding legal regulations, concerning in particular
remuneration, working hours, leave entitlements, health and safety
standards and decent treatment'\autocite[10]{european_union_for_fundamental_rights_severe_2015}. In the operations and supply chain management literature,
interest in businesses' respect for these kinds of employee labour rights began with
studies focused upon labour rights transgressions related to risk
reduction and risk communication and how to improve
employees' health and safety \autocite{chinander_aligning_2001,wolf_operationalizing_2001}. More recently, 
serious labour rights abuses have come to the fore with studies
examining the challenges of severe labour exploitation under the umbrella term `modern slavery' \autocite{gold_modern_2015,new_modern_2015,benstead_horizontal_2018,stevenson_modern_2018}. While this literature
offers important insights into severe forms of labour exploitation, particularly
in global supply chains, this and the wider social sustainability literature has
been criticised for its de-humanised approach to the understanding
of workers and their working conditions \autocite{soundararajan_humanizing_2021}. While at least one current
project seeks to examine the phenomenon of worker voice in factory
settings  \autocite{leverhulme_trust_research_2022}, there appears to be little attention paid to severe forms of labour exploitation from the workers' perspective in the private sphere. Nowhere are the realities of individual workers' experiences of employer exploitation brought into sharper relief than in the setting of domestic
work in private households.

The authors of the Global Slavery Index estimate that there are
seventy-six million people employed in domestic work worldwide
\autocite{international_labour_organization_global_2022}).
According to \textcite{bonnet_domestic_2022}, eighty percent of this domestic work is unregulated and informal. Labour
exploitation has been identified as an extensive global problem within
the sector, with domestic work identified as one of five private sector
groupings which contribute the most to forced labour. Defined in the ILO
Forced Labour Convention, 1930 No.29, forced or compulsory labour is
`all work or service which is exacted from a person under the threat of
a penalty and for which the person has not offered himself or herself
voluntarily' \autocite{ilo_what_2024}. Seventy-six
percent of domestic workers are women, and these workers represent four
percent of the total female workforce \autocite{international_labour_organization_global_2022}. Indeed, women in forced labour are much more likely to be in domestic
work than in any other occupation \autocite{international_labour_organization_global_2022}.
The ILO suggest that female domestic workers may be coerced through
non-payment of wages; abuse of vulnerability; subjected to physical and
sexual violence or experience threats against their family members. Such
severe forms of labour exploitation may be present alongside other,
perhaps less severe but equally illegal, practices which constitute
various forms of labour abuse. The criminalisation of both labour
exploitation and abuse in a domestic setting has developed in recent
times, with legislation enacted in the United Kingdom (UK), Europe,
Australia and Norway to criminalise such severe exploitation under the
term `modern slavery'. However, even where modern slavery laws are in
place, reliance on traditional, inspection-led, approaches to detection
designed primarily to ensure labour rights compliance within communal
workplaces such as factories mean that the number of reported cases of
labour exploitation in private dwellings may well severely underestimate
actual exploitation levels. This article aims to contribute to a more
nuanced understanding of how survey techniques may be developed to
improve understanding of the realities of workers' conditions by
demonstrating the use of a statistically robust estimation of the nature
and proportion of labour exploitation and abuse among domestic workers
in the UK. This setting was chosen due to long-standing national
legislation criminalising modern slavery introduced to the UK in 2015.
Despite, or perhaps because of this legislation, in recent years the
number of potential victims entering the UK's National Referral
Mechanism (NRM), a scheme which provides government support for those
suspected to be modern slavery survivors, has continued to increase.
Nineteen thousand, one hundred and twenty-five potential victims were
recorded in 2024: the highest annual figure since the NRM began
\autocite{home_office_modern_2025}. In 2024, for the
first time the number of cases of potential modern slavery among females
handled by the charity Unseen, who run the UK's modern slavery helpline,
were more prevalent than those among men 
(\cite{carter_women_2025}). Despite these worrying
headline statistics, and the persistence of specific concerns about high
levels of exploitation among domestic workers in the grey literature
\autocite{kalayaan_new_2008, mantouvalou_modern_2016, latin_american_womens_rights_service_behind_2023}, to our knowledge no-one has yet estimated the nature and extent of labour exploitation and abuse that may exist among domestic workers in the UK.

In contrast to overseas factory workers in globally dispersed, product,
supply chains, many service workers engaged in domestic work have
migrated to work in the UK. These transnational workers enter on
restricted visas where their employment---and their right to remain in
the country---is tied to their continuing employment. It is now ten
years since the UK's Modern Slavery Act was enacted. During its passage
through parliament, those advocating for the rights of domestic workers
were successful in expanding the final category boundaries of the
legislation to include, in Section 53, the specific definition of
(overseas) domestic workers as modern slavery victims 
\autocite{caruana_boundaries_2025}. These transnational
migrants are at particular risk of exploitation due to regulatory visa
restrictions and intersecting structural issues related to their gender,
the relative isolation of domestic work and a lack of supportive social
networks. This can mean that they fall out of legal migratory status.
Due to the social stigma attached to such illegal working, transnational
workers remaining in the UK without the right to work may be considered
a hidden, hard-to-reach, population. Extracting a sample of domestic
workers which includes this group raises difficulties when trying to
employ the normal statistical sampling methods considered necessary for
robust prevalence estimation. Perhaps due to these sampling
difficulties, we know relatively little about the nature of labour
exploitation among this particularly at-risk group of workers.
Fortunately, there has been significant interest in the development of
alternative methods for prevalence estimation which include such
hard-to-reach groups, with many scholars advocating and developing the
use of respondent-driven sampling (RDS) techniques to support
statistically robust estimators.

In this paper, we make two specific contributions to the operations and
supply chain management literature. First, we demonstrate the use of RDS
coupled with Network Scale-up Methods (N-SUM) to reach and sample
respondents' views of their working conditions among these,
predominantly female, transnational migrant domestic workers. We use the
data we obtain from these respondents to show how such a survey can be
used to estimate the proportion of workers experiencing labour
exploitation. Second, we begin to capture the nature and extent of
modern slavery as voiced by domestic service workers thereby, we
believe, expanding the nascent literature on worker voice which has, in
the main, focussed primarily upon factory workers 
\autocite{stephens_theorising_2024}. These contributions
not only extend our understanding of the risks of labour exploitation
and abuse among service workers engaged in domestic settings but also
show how it is possible to shed light on the severity of the
individuals' experience of exploitation through the construction of a
novel risk index. The remainder of this paper is structured as follows.
our study in more detail, highlighting what is already known about the
current population of domestic workers in the UK and the conditions in
which they work. Next, we describe our research methods. We review the
development of the respondent-driven sampling (RDS) techniques we used
and explain why this sampling method is suitable for our study. We then
describe our survey methods, including how we designed our survey
instrument, contacted our sample seeds and analysed our data. We then
present and discuss our findings, detailing the proportional estimate
that we calculated and the risk index we constructed. In our discussion,
we expand upon the implications of our findings for government policy,
enforcement practices and further research, including how these methods
may be used in future studies of labour exploitation in other sectoral
and geographic contexts. The limitations of our study are outlined,
before, finally, we conclude our article.

Labour exploitation has been defined as `work situations that deviate
significantly from standard working conditions as defined by legislation
or other binding legal regulations, concerning in particular
remuneration, working hours, leave entitlements, health and safety
standards and decent treatment'
\autocite[10]{european_union_for_fundamental_rights_severe_2015}. In the
operations and supply chain management literature, businesses' respect
for these kinds of employee labour rights began with studies focused
upon labour rights transgressions related to risk reduction and risk
communication and considered how to improve employees' health and safety
\autocite{chinander_aligning_2001,wolf_operationalizing_2001}. Recently,
more serious labour rights abuses have come to the fore with studies
examining the challenges of severe labour abuse under the umbrella term
`modern slavery'
\autocite{gold_modern_2015,new_modern_2015,benstead_horizontal_2018,stevenson_modern_2018}.
While this literature offers important insights into these severe forms
of labour exploitation in global supply chains, the wider social
sustainability literature has been criticised for taking a de-humanised
approach to the understanding of workers and their working conditions
(\textcite{soundararajan_humanizing_2021}). Perhaps as a result, little
attention has been paid to workers' experiences of severe forms of
labour exploitation in the literature to date. While recent projects
seek to examine the phenomenon of worker voice in factory settings
(\textcite{leverhulme_trust_research_2022}), nowhere are the realities
of individual workers' experiences of employer exploitation brought into
sharper relief than in the setting of domestic work in private
households.

The authors of the Global Slavery Index estimate that there are
seventy-six million people employed in domestic work worldwide
(\textcite{international_labour_organization_global_2022}).\footnote{CE:
  IS THIS THE CORRECT REFERENCE?} According to
\textcite{bonnet_domestic_2022}, eighty percent of this domestic work is
unregulated and informal. Labour exploitation has been identified as an
extensive global problem within the sector, with domestic work
identified as one of five private sector groupings which contribute the
most to forced labour. Defined in the ILO Forced Labour Convention, 1930
No.29, forced or compulsory labour is `all work or service which is
exacted from a person under the threat of a penalty and for which the
person has not offered himself or herself voluntarily'
(\textcite{international_labour_organization_global_2022}). Seventy-six
percent of domestic workers are women, and these workers represent four
percent of the total female workforce
(\textcite{international_labour_organization_global_2022}). Indeed,
women in forced labour are much more likely to be in domestic work than
in any other occupation
\autocite{international_labour_organization_global_2022}. The ILO
suggest that female domestic workers may be coerced through non-payment
of wages; abuse of vulnerability; subjected to physical and sexual
violence or experience threats against their family members. Such severe
forms of labour exploitation may be present alongside other, perhaps
less severe but equally illegal, practices which constitute various
forms of labour abuse. The criminalisation of both labour exploitation
and abuse in a domestic setting has developed in recent times, with
legislation enacted in the United Kingdom (UK), Europe, Australia and
Norway to criminalise such severe exploitation under the term `modern
slavery'. However, even where modern slavery laws are in place, reliance
on traditional, inspection-led, approaches to detection designed
primarily to ensure labour rights compliance within communal workplaces
such as factories mean that the number of reported cases of labour
exploitation in private dwellings may well severely underestimate actual
exploitation levels. This article aims to contribute to a more nuanced
understanding of how survey techniques may be developed to improve
understanding of the realities of workers' conditions by demonstrating
the use of a statistically robust estimation of the nature and
proportion of labour exploitation and abuse among domestic workers in
the UK. This setting was chosen due to long-standing national
legislation criminalising modern slavery introduced to the UK in 2015.
Despite, or perhaps because of this legislation, in recent years the
number of potential victims entering the UK's National Referral
Mechanism (NRM), a scheme which provides government support for those
suspected to be modern slavery survivors, has continued to increase.
Nineteen thousand, one hundred and twenty-five potential victims were
recorded in 2024: the highest annual figure since the NRM began
\autocite{home_office_modern_2025}. In 2024, for the first time the
number of cases of potential modern slavery among females handled by the
charity Unseen, who run the UK's modern slavery helpline, were more
prevalent than those among men (\textcite{carter_women_2025}). Despite
these worrying headline statistics, and the persistence of specific
concerns about high levels of exploitation among domestic workers in the
grey literature (\textcite{kalayaan_new_2008};
\textcite{mantouvalou_modern_2016};
\textcite{latin_american_womens_rights_service_behind_2023}), to our
knowledge no-one has yet estimated the nature and extent of labour
exploitation and abuse that may exist among domestic workers in the UK.

In contrast to overseas factory workers in globally dispersed, product,
supply chains, many service workers engaged in domestic work have
migrated to work in the UK. These transnational workers enter on
restricted visas where their employment---and their right to remain in
the country---is tied to their continuing employment. It is now ten
years since the UK's Modern Slavery Act was enacted. During its passage
through parliament, those advocating for the rights of domestic workers
were successful in expanding the final category boundaries of the
legislation to include, in Section 53, the specific definition of
(overseas) domestic workers as modern slavery victims
(\textcite{caruana_boundaries_2025}). These transnational migrants are
at particular risk of exploitation due to regulatory visa restrictions
and intersecting structural issues related to their gender, the relative
isolation of domestic work and a lack of supportive social networks.
This can mean that they fall out of legal migratory status. Due to the
social stigma attached to such illegal working, transnational workers
remaining in the UK without the right to work may be considered a
hidden, hard-to-reach, population. Extracting a sample of domestic
workers which includes this group raises difficulties when trying to
employ the normal statistical sampling methods considered necessary for
robust prevalence estimation. Perhaps due to these sampling
difficulties, we know relatively little about the nature of labour
exploitation among this particularly at-risk group of workers.
Fortunately, there has been significant interest in the development of
alternative methods for prevalence estimation which include such
hard-to-reach groups, with many scholars advocating and developing the
use of respondent-driven sampling (RDS) techniques to support
statistically robust estimators.

In this paper, we make two specific contributions to the operations and
supply chain management literature. First, we demonstrate the use of RDS
coupled with Network Scale-up Methods (N-SUM) to reach and sample
respondents' views of their working conditions among these,
predominantly female, transnational migrant domestic workers. We use the
data we obtain from these respondents to show how such a survey can be
used to estimate the proportion of workers experiencing labour
exploitation. Second, we begin to capture the nature and extent of
modern slavery as voiced by domestic service workers thereby, we
believe, expanding the nascent literature on worker voice which has, in
the main, focussed primarily upon factory workers
(\textcite{stephens_theorising_2024}). These contributions not only
extend our understanding of the risks of labour exploitation and abuse
among service workers engaged in domestic settings but also show how it
is possible to shed light on the severity of the individuals' experience
of exploitation through the construction of a novel risk index. The
remainder of this paper is structured as follows. First, we describe our
conceptual framework and introduce the context of our study in more
detail, highlighting what is already known about the current population
of domestic workers in the UK and the conditions in which they work.
Next, we describe our research methods. We review the development of the
respondent-driven sampling (RDS) techniques we used and explain why this
sampling method is suitable for our study. We then describe our survey
methods, including how we designed our survey instrument, contacted our
sample seeds and analysed our data. We then present and discuss our
findings, detailing the proportional estimate that we calculated and the
risk index we constructed. In our discussion, we expand upon the
implications of our findings for government policy, enforcement
practices and further research, including how these methods may be used
in future studies of labour exploitation in other sectoral and
geographic contexts. The limitations of our study are outlined, before,
finally, we conclude our article.

Conceptualising Labour Exploitation and the Degree of Risk

Modern slavery has been criticised by some for its overly extensive
scope: encapsulating a broad range of divergent sub-categories of
exploitation (\textcite{oconnell_davidson_margins_2015};
\textcite{gutierrez-huerter_o_change_2023}). For this reason, we used
the International Labour Organization's
(\textcite{ILO11-indicators})\footnote{CE: No idea if this is the
  correct reference -- it is my best guess.} `Indicators of Forced
Labour' to identify the potential for severe labour exploitation and as
a basis for the quantification of our labour exploitation and abuse risk
index. The ILO identify eleven indicators designed to help understand
how forced labour arises and how it affects victims. These indicators
include: abuse of vulnerability; deception; restriction of movement;
isolation; physical and sexual violence; intimidation and threats;
retention of identity documents; withholding of wages; debt bondage;
abusive working and living conditions and excessive overtime. According
to the ILO, the presence of a single indicator in any given situation
may in some cases imply the existence of forced labour. However, it also
suggests that in other cases it may be necessary to look for several
indications which, taken together, may point to a case of forced labour.
We seek to refine this statement through the construction of a composite
index by which means a degree of risk related to the likelihood of a
domestic worker experiencing this most severe form of exploitation may
be distinguished from the likely occurrence of less severe, though
similarly illegal, forms of labour abuse.

\section{Evaluating the Degree of
Risk}\label{evaluating-the-degree-of-risk}

The study of risk management has a long tradition in operations and
supply chain management. Initially, the risks under consideration were
primarily related to ensuring continuity of the supply of goods and
services (see for example, \textcite{juttner_supply_2003}). Beginning
with \textcite{anderson_critical_2006} and
\textcite{anderson_sustainability_2009}, however, a literature stream of
sustainability-related supply chain risk management developed related
specifically to the risks associated with the environment and social
justice. A normative consensus related to the main stages of supply
chain risk management has developed in the literature, with a five-stage
sequential model typically presented. There have also been empirical
studies of risk management within various industrial supply chains in
the United States and India (\textcite{tarei_hybrid_2018};
\textcite{dellana_scale_2021}), including the quantification of a risk
index for the petroleum supply chain (\textcite{tarei_hybrid_2018}).
Yet, while these authors recognize the need for responsible management
and its effect on societal values, in line with other literature in the
field they view risk from the perspective of the corporate supply chain
rather than examining the risk of harm to the worker.

In our study, we conceptualise the risk of labour exploitation from the
workers' perspective. We conceive severe forms of labour exploitation
such as forced labour as one end of a spectrum ranging from illegal
employment practices that constitute labour abuse, such as wage payments
below legal minimum wage levels and health and safety violations,
through to the likelihood of criminal exploitation recognized in the UK
as modern slavery. Our assessment of this personal risk permits a degree
of risk to be assigned to various clusters of forced labour indicators
with the more indicators present, the stronger the likelihood that the
working conditions may be considered exploitative. Our approach,
therefore, includes, but goes beyond, assessing the likelihood of forced
labour by simply quantifying the proportion of survivors entering the
UK's National Referral Mechanism (NRM): a government system for survivor
support set up to identify whether there are positive grounds for the
identification of Modern Slavery. In our method, an NRM referral is used
as the strongest indicator of modern slavery risk, with lesser risks
assessed according to the degree to which cumulative indicators of
forced labour are reported.

Case Setting: Labour Exploitation Risk Among Transnational Migrant
Domestic Workers In The UK

Domestic work forms part of a broader industrial category of Personal
and Household Service work (PHS). Work in this category includes those
employed in `social work activities without accommodation' and
`activities of households as employers of domestic personnel'
(\textcite{european_commission_staff_2012}). In 2017, an estimated
980,000 people were engaged in PHS work in the UK
(\textcite{manoudi_analysis_2018}). \textcite{manoudi_analysis_2018}
highlight that the PHS sector is dominated by women and migrants, with
many undeclared foreign workers. Detailed statistics related to the
country of origin of domestic workers migrating to work in PHS in the UK
are difficult to isolate before 2019. Since that time, annual migration
has fluctuated -- falling sharply in 2021 due in part to the COVID-19
pandemic, before later rising again above pre-pandemic levels. In the
year to December 2022, the UK Home Office reported that it had issued
18,533 Overseas Domestic Worker visas (\textcite{home_office_why_2023}).
These domestic workers came from various countries in South America and
Asia, including many from the Philippines.

In 2023, \textcite{strauss_britain_2023} reported a big shift in the
source countries of migrants arriving in the UK on the Overseas Domestic
Worker and other types of worker visas. Transnational domestic workers
from the Philippines and India accounted for the single largest number
of applications granted (10,186 and 3,858 visas respectively), followed
by smaller, but still significant, numbers of workers arriving from
Bangladesh (465), Nigeria (446), Sri Lanka (444), Egypt (422), and
Ethiopia (285). In the same period, smaller numbers of visa applications
to work as a domestic worker in the UK were also accepted from workers
from other source countries including, but not limited to, the Sudan,
Nepal, Ghana, Kenya, Lebanon, Eritrea, Iran, Turkey, Yemen, Malaysia,
Thailand, and Morocco. This post-Brexit increase in the diversity of
source countries from which transnational workers are drawn makes a more
detailed analysis of the risk of labour exploitation in the sector both
timely and more urgent.

There is a long history of reports of exploitation in the domestic work
sector in the UK. In 2008, the civil society organisation Kalayaan,
which was formed to campaign for the formal recognition of migrant
domestic workers' rights in the UK, reported on the impact of proposed
changes to the UK immigration system on migrant domestic workers
(\textcite{kalayaan_new_2008}). Their report highlights government
recognition of documented and unacceptable levels of abuse and
exploitation among domestic workers in the UK as early as 1996. At this
stage, new policies, including the development of a specialised visa
allowing domestic workers to change employer during their stay were
introduced. However, in 2012, these visa conditions were modified, tying
domestic workers to a single employer and restricting the length of time
that they are permitted to remain in the country to a period of six
months (\textcite{gower_calls_2016}). Overseas domestic worker visa
holders are now, again, permitted to change employers, but not to apply
to renew their six-month long visa unless they receive a positive
`Conclusive Grounds' decision related to exploitation considered to be
modern slavery through the UK's National Referral Mechanism (NRM)
(\textcite{romero_blueprint_2025}).

These reports highlight the underlying reasons for migrant domestic
workers' vulnerability, including workers' relative desperation for
work; their lack of social ties; unfamiliarity with English language and
culture; long working hours; lack of knowledge of their legal rights; a
lack of oversight of the private home as a workplace; their work forming
part of the informal economy; their reliance on their employer for
permission to work in the UK; and their lack of recourse to public
funds. As a result, migrant domestic workers are vulnerable to abuse
ranging from minor breaches of employment and health and safety law, to
physical and sexual violence, slavery, forced labour and trafficking.

That these conditions may persist is evidenced by a report from another
civil society organisation, the Latin American Women's Rights Service,
which describes the results from twelve in-depth interviews with Latin
American domestic workers in the UK. This report depicts high levels of
isolation, exploitation and abuse including a failure by employers to
provide written contracts or payslips; breaches of verbal agreements; a
requirement to perform different tasks from those indicated during
recruitment; increasing working hours with little or no time off;
excessive work days; a lack of paid holiday; many domestic workers not
registered with a GP; sexual harassment in the workplace; verbal or
physical abuse; employer surveillance; a lack of opportunity to change
working conditions; isolation and fear of seeking help; and high
reported levels of trafficking for labour exploitation
(\textcite{latin_american_womens_rights_service_behind_2023}).

Against this backdrop, we used respondent driven sampling (RDS) as a
sampling technique to recruit and survey domestic workers in the UK
about the working conditions they were experiencing to estimate the
nature and scale of abuse and exploitation based upon reports of their
conditions by domestic workers themselves.

Research Methods

{[}Scott- do feel free to edit as you see fit{]}

\section{Respondent-Driven Sampling (RDS) And Survey
Method}\label{respondent-driven-sampling-rds-and-survey-method}

Comprehensive descriptions and literature reviews of the development and
use of RDS to estimate the population size of a hidden population are
available elsewhere (\textcite{heckathorn_comment_2011};
\textcite{gile_methods_2018}). Suffice it to say, the possibilities of
the use of a one-wave snowball sampling to allow researchers to obtain a
sample of personal networks was posited by
\textcite{frank_estimating_1994}. Following the identification of a set
of original sample members known as seeds,
\textcite{heckathorn_respondent-driven_1997};
\textcite{heckathorn_respondent-driven_2002} advocate the use of a
double incentive to recompense participants not only for their
involvement, but also for their recruitment of further participants in
subsequent `waves' of participation by drawing upon the social ties
through which members of the hidden population are connected to each
other.

The typical number of original sample seeds is between two and ten:
chosen as heterogeneously as possible (\textcite{gile_methods_2018}).
Though they may be subject to both systematic and non-systematic errors,
the use of snowballing methods for the study of hidden populations, with
the support of monetary or symbolic rewards, has been advocated as a way
of creating robust recruitment embodying diversity in characteristics
such as ethnicity, gender and geographical location
(\textcite{heckathorn_respondent-driven_1997};
\textcite{heckathorn_respondent-driven_2002}). In these papers,
Heckathorn advances the development of RDS to include self-reported
network size as a population estimator and bootstrapping techniques to
support the development of an estimator's confidence intervals, an
approach that has since been refined by others
(\textcite{gile_network_2015}). Such developments derive a new class of
indicators for the population mean and define a corresponding bootstrap
method to estimate the errors in RDS. The resulting `network working
model' permits the individual's connectedness in the network to be
tested, while reducing bias with respect to the composition of the
seeds. Snowball sampling is based upon the initial recruitment of the
original sample selection by means of convenience. RDS also takes a
non-random approach to seed selection, but relies upon the social
network structure that exists between participants to produce a
non-probabilistic sample (\textcite{goodman_comment_2011}). Incentive
structure is important---though this weakness is not a feature of our
target hidden population, some researchers have identified that younger
men with higher socio-economic status are less likely to participate
(\textcite{mccreesh_respondent_2013}). Perhaps of more concern, RDS has
been described as a risky strategy since researchers cannot be sure
whether enough respondents have been recruited through subsequent waves
to eliminate bias within the original sample members
(\textcite{vincent_estimating_2017}).

RDS has been widely used to sample a variety of hidden populations,
including HIV prevalence, rape and client-initiated gender-based
violence among sex workers (\textcite{mccreesh_evaluation_2012};
\textcite{schwitters_prevalence_2012}). While the RDS method has proved
limited when seeking to provide population heterogeneity by geographical
location (\textcite{mccreesh_evaluation_2011}), where these population
features are of lesser importance, such methods have been used
successfully. RDS methods have been used to survey other migrant
populations (\textcite{tyldum_surveying_2021}), while such network-based
referrals have been described as the only viable method to reach many
types of labour trafficking victims (\textcite{zhang_measuring_2012})
and have been used to research exploitation among low-wage workers in
three American cities (\textcite{bernhardt_broken_2009}); a study of
labour trafficking in migrant communities in the city of San Diego
(\textcite{vincent_estimating_2017}); examination of the worst forms of
child labour in the Indian state of Bihar
(\textcite{zhang_victims_2019}); and the commercial sexual exploitation
of children in Nepal (\textcite{jordan_overcoming_2020}).

In the following section, we describe our methods, including how we
designed our survey, contacted our sample seeds, and analysed our data.
Our approach can best be described as Web-based RDS
(\textcite{wejnert_web-based_2008}). We designed a web survey using the
JISC online survey interface, suitable for our respondents to complete
via a mobile phone. Composite measures to quantify the extent to which
respondents were at risk of labour exploitation, including severe forms
of exploitation such as forced labour, were constructed from existing
exploitation typologies, notably the ILO's Indicators
(\textcite{ILO11-indicators}). The survey consisted of these 11
composite indicators and included questions related to domestic workers'
level of job satisfaction, employment conditions, and demographic data
such as nationality, age, and gender. The main survey was conducted in
the five months between February and July 2023.

\section{Initial Sample Selection}\label{initial-sample-selection}

We selected our first wave of participants nonrandomly by convenience
sampling. Mobile phone numbers were used both to identify seed
participants and to act as a unique identifier for those whom they
referred. To avoid sample homophily, original sample members were
selected from three distinct domestic worker communities. This was
facilitated by civil society organisations who represented distinct
domestic worker communities. One was an exclusively online community of
transnational domestic workers working in the UK, the second represented
UK domestic workers of Filipino origin, and the third drew its
membership from the Latin American community of domestic workers, also
in the UK. Along with other academics with expertise in exploitation
within domestic work, representatives from these three organisations
also contributed to survey question design and facilitated the piloting
of an initial version of the survey (which was translated and made
available in four languages: English, Spanish, Tagalog, and Portuguese)
to selected domestic workers within each community.

\section{Survey Incentives: Incentive Design and Participation
Verification}\label{survey-incentives-incentive-design-and-participation-verification}

A double incentive scheme rewarded respondents both for completing the
questionnaire and for each referral who went on to engage with the
survey. The challenge of incentive design is to set the incentive at a
level that adequately rewards respondents' time and participation, but
that also avoids the risk of fraudulent participation due to too high a
monetary gain (\textcite{jordan_overcoming_2020}). A sum of £10 was
provided for survey completion with a further £5 for each successful
nomination. While respondents were asked to nominate up to 10 domestic
workers within their existing social network, it was the first three of
these from whom participation was requested in subsequent waves. This
approach is akin to the use of vouchers in face-to-face studies as
advocated by \textcite{thompson_new_2020}.

The ethical and practical issues related to the design and effective use
of incentives for RDS among vulnerable populations has been much
discussed in the literature; see, for example,
\textcite{wang_respondent-driven_2005};
\textcite{abdul-quader_effectiveness_2006};
\textcite{singer_incentives_2006}; \textcite{dejong_ethical_2009};
\textcite{semaan_ethical_2009}; \textcite{brunovskis_untold_2010};
\textcite{semaan_time-space_2010}; \textcite{platt_adapting_2015},
including the specificities of incentive use within web-based surveys
(\textcite{cobanoglu_effect_2003}). Following the principles of lottery
use established by \textcite{brown_you_2006} and
\textcite{laguilles_can_2011}, we also designed our survey to encourage
the maximum extent of participation by entering all respondents
completing the questionnaire into a free prize draw for £150. Research
suggests that a high lottery provides the most cost-effective incentive
for obtaining complete responses
(\textcite{gajic_cost-effectiveness_2012}). While using incentives to
encourage participation would seem to be desirable, it is worth noting
the potential downside of respondents fabricating responses to increase
their remuneration (\textcite{robinson_sampling_2014}). To minimise this
risk, mobile phone numbers for each respondent and those whom they
referred were collated, and each of these numbers was called by one of
the authors of the paper to ascertain the veracity of the respondent as
a migrant domestic worker.

Data Analysis

\section{Descriptive Statistics}\label{descriptive-statistics}

In total, we received completed online surveys from 97 respondents. Of
these respondents, 90 identified themselves as transnational migrants.
Forty-five percent regarded themselves as self-employed, 39\% identified
themselves as employees, and 16\% categorised their employment status as
that of a worker.

Of the 97 respondents, 64 (66\% of the total), and the largest single
nationality group, reported that they had a Filipina background. Other
nationalities represented included Dominican, Brazilian, Spanish,
Colombian, Bolivian, Venezuelan, Cuban, and Panamanian. Female domestic
workers made up 97\% of the sample, with 3\% of the sample comprised of
male domestic workers. The age structure of the domestic workers was
skewed towards those over 45 years old, with such workers representing
over half of the sample (see Table 1).

\section{Network Structure}\label{network-structure}

GRAPH HERE.

Findings

\section{Point Estimation and Confidence
Intervals}\label{point-estimation-and-confidence-intervals}

\section{Risk Index}\label{risk-index}

Discussion

{[}Scott to add and revise{]}

\section{Implications for Policy}\label{implications-for-policy}

The UK Government has proved reluctant to respond to calls to remove the
restrictive, tied, visa conditions currently in force for those migrant
workers working in the UK on the Overseas Domestic Workers visa
(\textcite{gower_calls_2016}). Maintaining these restrictive conditions
prevents the ratification in the UK of C189, the International
Convention for Domestic Workers (\textcite{ILO11-indicators}). If the
estimates resulting from our study are correct, these visa conditions
place migrant domestic workers at significant risk of serious forms of
labour exploitation including, in its most severe form, exploitation
that exhibits the characteristics of forced labour---legally considered
a form of modern slavery.

To reduce the vulnerability of transnational domestic workers to
this---and other---forms of labour exploitation, we urge policy-makers
to reconsider these discriminatory visa conditions and offer the same
freedoms to domestic workers that are enjoyed by other groups of workers
under UK law.

In addition, given the vulnerabilities experienced by workers due to the
private nature of the workplace, we would urge the UK government to
consider the regulation of domestic worker employers.

Finally, given the stigma and very real danger of deportation of those
migrant domestic workers who may have fallen out of legal migration
status, our evidence suggests that there is an urgent need for the UK
Government to enforce a firewall between immigration control and labour
exploitation if the true scale of abuse is to be made visible and the
perpetrators brought to justice.

\section{Implications for Practice}\label{implications-for-practice}

The UK Visa and Immigration service already offers rights-based training
to migrant domestic workers via UK embassies in certain source
countries. To reduce migrant domestic workers vulnerabilities, we
advocate the expansion of this training both to include explicit
training related to employment and labour rights within the UK and to
the rapidly expanding range of new source countries from where migrant
domestic workers are now drawn.

\section{Further Research}\label{further-research}

We believe that web-RDS combined with statistical estimators such as
NSUM offers an important method for the capture and comparison of
relative proportions of labour exploitation and abuse in sectors within
and beyond the UK. Network scale up methods, and potential enhancements
such as Generalised network scale up estimators offer to enhance
understanding, not least within operations and supply chain management
research, of the extent of labour exploitation in different sectors and
across industries.

\section{Limitations of the Study}\label{limitations-of-the-study}

As with any empirical research, our study is subject to limitations. In
terms of nationality, our sample is not representative of the
demographics of those domestic workers employed on Overseas Domestic
Worker Visas in 2022 the UK. Due to the increasing number of workers on
Overseas Domestic Work visas from the Indian sub-continent, attempts
were made also to seed respondents from this community. This proved
difficult, with anecdotal information suggesting that domestic workers
from this community rarely had access to a personal mobile phone. It is
not therefore possible to infer the nature and extent of labour
exploitation within this sub-section of the domestic worker population.

As the network structure of our sample demonstrates, even with a
well-designed incentive scheme it proved difficult to recruit
respondents from these communities of domestic workers in subsequent
sampling waves in the time available. Most of our respondents are
therefore original sample members draw from the three domestic worker
communities used to seed the survey.

Conclusion

\newpage

\newpage

\printbibliography

\newpage

References

\printbibliography[heading=none]

\newpage

\appendix

Appendix A

Population Parameters

Use \textasciitilde980,000 as UK domestic worker population estimate (EU
data) Use 44,360 as NRM adult referrals baseline Address treatment of
``don't know'' responses and zero network size claims Consider separate
analysis for Filipino subgroup

\section{The Survey}\label{the-survey}

(From codebook)

Compare corresponding survey questions between RDS and NSUM methods:

\begin{itemize}
\tightlist
\item
  Q70/Q71 (document withholding)
\item
  Q39+Q42/Q43 (pay issues)
\item
  Q45+Q47+Q48/Q49 (abuse/threats)
\item
  Q61+Q62/Q64 (excessive hours)
\item
  Q78/Q79 (access to help)
\end{itemize}

Risk Index Implementation

\begin{itemize}
\tightlist
\item
  Clean coding for 13 risk categories with proper weightings
\item
  NRM referral (0.35)
\item
  Forced labor indicators (0.55 total)
\item
  Below minimum wage (0.10)
\end{itemize}

Appendix B

\section{Executive Summary}\label{executive-summary}

This analysis estimates the prevalence of modern slavery among domestic
workers in the UK using multiple RDS methodologies. We examine two
primary indicators across various population size assumptions and
estimation techniques.

\textbf{Key Findings:} - Below minimum wage prevalence (Q36): X.X\% -
Y.Y\% (95\% CI) - NRM referral experience (Q80): A.A\% - B.B\% (95\%
CI)\\
- Population size estimates: 980,000 - 1.74M domestic workers

\section{Sample Characteristics}\label{sample-characteristics}

\subsection{Recruitment Network
Structure}\label{recruitment-network-structure}

\begin{table}

\caption{\label{tbl-network}RDS Sample Network Characteristics}

\centering{

\begin{Shaded}
\begin{Highlighting}[]
\NormalTok{network\_stats }\OtherTok{\textless{}{-}} \FunctionTok{tibble}\NormalTok{(}
  \AttributeTok{Characteristic =} \FunctionTok{c}\NormalTok{(}\StringTok{"Total Sample Size"}\NormalTok{, }\StringTok{"Recruitment Waves"}\NormalTok{, }\StringTok{"Average Degree"}\NormalTok{, }
                    \StringTok{"Median Degree"}\NormalTok{, }\StringTok{"Seeds"}\NormalTok{, }\StringTok{"Longest Chain"}\NormalTok{, }\StringTok{"Mean Chain Length"}\NormalTok{),}
  \AttributeTok{Value =} \FunctionTok{c}\NormalTok{(}
    \FunctionTok{nrow}\NormalTok{(rd.dd),}
    \FunctionTok{max}\NormalTok{(rd.dd}\SpecialCharTok{$}\NormalTok{wave, }\AttributeTok{na.rm =} \ConstantTok{TRUE}\NormalTok{),}
    \FunctionTok{round}\NormalTok{(}\FunctionTok{mean}\NormalTok{(rd.dd}\SpecialCharTok{$}\NormalTok{numRef, }\AttributeTok{na.rm =} \ConstantTok{TRUE}\NormalTok{), }\DecValTok{1}\NormalTok{),}
    \FunctionTok{round}\NormalTok{(}\FunctionTok{median}\NormalTok{(rd.dd}\SpecialCharTok{$}\NormalTok{numRef, }\AttributeTok{na.rm =} \ConstantTok{TRUE}\NormalTok{), }\DecValTok{1}\NormalTok{),}
    \FunctionTok{sum}\NormalTok{(rd.dd}\SpecialCharTok{$}\NormalTok{recruiter.id }\SpecialCharTok{==} \SpecialCharTok{{-}}\DecValTok{1}\NormalTok{),}
    \StringTok{"TBD"}\NormalTok{, }\CommentTok{\# Calculate from recruitment chains}
    \StringTok{"TBD"}  \CommentTok{\# Calculate average chain length}
\NormalTok{  )}
\NormalTok{)}

\NormalTok{network\_stats }\SpecialCharTok{\%\textgreater{}\%}
  \FunctionTok{gt}\NormalTok{() }\SpecialCharTok{\%\textgreater{}\%}
  \FunctionTok{tab\_header}\NormalTok{(}\AttributeTok{title =} \StringTok{"Network Characteristics"}\NormalTok{) }\SpecialCharTok{\%\textgreater{}\%}
  \FunctionTok{fmt\_number}\NormalTok{(}\AttributeTok{columns =}\NormalTok{ Value, }\AttributeTok{decimals =} \DecValTok{1}\NormalTok{, }\AttributeTok{use\_seps =} \ConstantTok{TRUE}\NormalTok{)}
\end{Highlighting}
\end{Shaded}

}

\end{table}%

\subsection{Demographic Composition}\label{demographic-composition}

\begin{Shaded}
\begin{Highlighting}[]
\CommentTok{\# Create nationality breakdown}
\NormalTok{nationality\_summary }\OtherTok{\textless{}{-}}\NormalTok{ rd.dd }\SpecialCharTok{\%\textgreater{}\%}
  \FunctionTok{group\_by}\NormalTok{(nationality\_cluster) }\SpecialCharTok{\%\textgreater{}\%}
  \FunctionTok{summarise}\NormalTok{(}
    \AttributeTok{n =} \FunctionTok{n}\NormalTok{(),}
    \AttributeTok{percent =} \FunctionTok{round}\NormalTok{(}\FunctionTok{n}\NormalTok{() }\SpecialCharTok{/} \FunctionTok{nrow}\NormalTok{(rd.dd) }\SpecialCharTok{*} \DecValTok{100}\NormalTok{, }\DecValTok{1}\NormalTok{),}
    \AttributeTok{.groups =} \StringTok{\textquotesingle{}drop\textquotesingle{}}
\NormalTok{  )}

\FunctionTok{ggplot}\NormalTok{(nationality\_summary, }\FunctionTok{aes}\NormalTok{(}\AttributeTok{x =}\NormalTok{ nationality\_cluster, }\AttributeTok{y =}\NormalTok{ percent, }\AttributeTok{fill =}\NormalTok{ nationality\_cluster)) }\SpecialCharTok{+}
  \FunctionTok{geom\_col}\NormalTok{() }\SpecialCharTok{+}
  \FunctionTok{geom\_text}\NormalTok{(}\FunctionTok{aes}\NormalTok{(}\AttributeTok{label =} \FunctionTok{paste0}\NormalTok{(n, }\StringTok{"}\SpecialCharTok{\textbackslash{}n}\StringTok{("}\NormalTok{, percent, }\StringTok{"\%)"}\NormalTok{)), }\AttributeTok{vjust =} \SpecialCharTok{{-}}\FloatTok{0.5}\NormalTok{) }\SpecialCharTok{+}
  \FunctionTok{scale\_fill\_viridis\_d}\NormalTok{() }\SpecialCharTok{+}
  \FunctionTok{theme\_minimal}\NormalTok{() }\SpecialCharTok{+}
  \FunctionTok{labs}\NormalTok{(}\AttributeTok{x =} \StringTok{"Nationality Cluster"}\NormalTok{, }\AttributeTok{y =} \StringTok{"Percentage"}\NormalTok{, }
       \AttributeTok{title =} \StringTok{"Sample Distribution by Nationality"}\NormalTok{) }\SpecialCharTok{+}
  \FunctionTok{theme}\NormalTok{(}\AttributeTok{legend.position =} \StringTok{"none"}\NormalTok{)}
\end{Highlighting}
\end{Shaded}

\section{RDS Estimation Results}\label{rds-estimation-results}

\subsection{Model-Assisted Estimates by Population
Size}\label{model-assisted-estimates-by-population-size}

\begin{table}

\caption{\label{tbl-ma-estimates}Model-Assisted Estimates by Population
Size and Seed Selection}

\centering{

\begin{Shaded}
\begin{Highlighting}[]
\CommentTok{\# Extract MA results and create comparison table}
\NormalTok{ma\_comparison }\OtherTok{\textless{}{-}} \FunctionTok{expand\_grid}\NormalTok{(}
  \AttributeTok{seed\_method =} \FunctionTok{c}\NormalTok{(}\StringTok{"sample"}\NormalTok{, }\StringTok{"random"}\NormalTok{, }\StringTok{"degree"}\NormalTok{),}
  \AttributeTok{pop\_size =} \FunctionTok{c}\NormalTok{(}\DecValTok{100000}\NormalTok{, }\DecValTok{953000}\NormalTok{, }\DecValTok{1000000}\NormalTok{, }\DecValTok{1500000}\NormalTok{),}
  \AttributeTok{indicator =} \FunctionTok{c}\NormalTok{(}\StringTok{"q36"}\NormalTok{, }\StringTok{"q80"}\NormalTok{, }\StringTok{"composite\_risk"}\NormalTok{)}
\NormalTok{) }\SpecialCharTok{\%\textgreater{}\%}
  \FunctionTok{rowwise}\NormalTok{() }\SpecialCharTok{\%\textgreater{}\%}
  \FunctionTok{mutate}\NormalTok{(}
    \AttributeTok{key =} \FunctionTok{paste0}\NormalTok{(seed\_method, }\StringTok{"\_"}\NormalTok{, pop\_size),}
    \AttributeTok{estimate =} \FunctionTok{ifelse}\NormalTok{(key }\SpecialCharTok{\%in\%} \FunctionTok{names}\NormalTok{(ma\_results),}
\NormalTok{                     ma\_results[[key]][[indicator]]}\SpecialCharTok{$}\NormalTok{estimate, }\ConstantTok{NA}\NormalTok{),}
    \AttributeTok{ci\_lower =} \FunctionTok{ifelse}\NormalTok{(key }\SpecialCharTok{\%in\%} \FunctionTok{names}\NormalTok{(ma\_results),}
\NormalTok{                     ma\_results[[key]][[indicator]]}\SpecialCharTok{$}\NormalTok{conf.int[}\DecValTok{1}\NormalTok{], }\ConstantTok{NA}\NormalTok{),}
    \AttributeTok{ci\_upper =} \FunctionTok{ifelse}\NormalTok{(key }\SpecialCharTok{\%in\%} \FunctionTok{names}\NormalTok{(ma\_results),}
\NormalTok{                     ma\_results[[key]][[indicator]]}\SpecialCharTok{$}\NormalTok{conf.int[}\DecValTok{2}\NormalTok{], }\ConstantTok{NA}\NormalTok{)}
\NormalTok{  ) }\SpecialCharTok{\%\textgreater{}\%}
  \FunctionTok{ungroup}\NormalTok{() }\SpecialCharTok{\%\textgreater{}\%}
  \FunctionTok{filter}\NormalTok{(}\SpecialCharTok{!}\FunctionTok{is.na}\NormalTok{(estimate))}

\CommentTok{\# Create formatted table}
\NormalTok{ma\_comparison }\SpecialCharTok{\%\textgreater{}\%}
  \FunctionTok{mutate}\NormalTok{(}
    \AttributeTok{pop\_size\_f =}\NormalTok{ scales}\SpecialCharTok{::}\FunctionTok{comma}\NormalTok{(pop\_size),}
    \AttributeTok{estimate\_ci =} \FunctionTok{sprintf}\NormalTok{(}\StringTok{"\%.3f (\%.3f, \%.3f)"}\NormalTok{, estimate, ci\_lower, ci\_upper)}
\NormalTok{  ) }\SpecialCharTok{\%\textgreater{}\%}
  \FunctionTok{select}\NormalTok{(seed\_method, pop\_size\_f, indicator, estimate\_ci) }\SpecialCharTok{\%\textgreater{}\%}
  \FunctionTok{pivot\_wider}\NormalTok{(}\AttributeTok{names\_from =}\NormalTok{ indicator, }\AttributeTok{values\_from =}\NormalTok{ estimate\_ci) }\SpecialCharTok{\%\textgreater{}\%}
  \FunctionTok{gt}\NormalTok{() }\SpecialCharTok{\%\textgreater{}\%}
  \FunctionTok{tab\_header}\NormalTok{(}\AttributeTok{title =} \StringTok{"Model{-}Assisted Estimates by Method"}\NormalTok{) }\SpecialCharTok{\%\textgreater{}\%}
  \FunctionTok{cols\_label}\NormalTok{(}
    \AttributeTok{seed\_method =} \StringTok{"Seed Selection"}\NormalTok{,}
    \AttributeTok{pop\_size\_f =} \StringTok{"Population Size"}\NormalTok{,}
    \AttributeTok{q36 =} \StringTok{"Below Min Wage"}\NormalTok{,}
    \AttributeTok{q80 =} \StringTok{"NRM Experience"}\NormalTok{, }
    \AttributeTok{composite\_risk =} \StringTok{"Composite Risk"}
\NormalTok{  ) }\SpecialCharTok{\%\textgreater{}\%}
  \FunctionTok{tab\_spanner}\NormalTok{(}\AttributeTok{label =} \StringTok{"Prevalence Estimates (95\% CI)"}\NormalTok{, }\AttributeTok{columns =} \FunctionTok{c}\NormalTok{(q36, q80, composite\_risk))}
\end{Highlighting}
\end{Shaded}

}

\end{table}%

\subsection{Traditional RDS Estimators
Comparison}\label{traditional-rds-estimators-comparison}

\begin{table}

\caption{\label{tbl-rds-comparison}RDS Estimator Comparison (N=980,000)}

\centering{

\begin{Shaded}
\begin{Highlighting}[]
\CommentTok{\# Extract RDS estimates for standard population size}
\NormalTok{rds\_980k }\OtherTok{\textless{}{-}}\NormalTok{ rds\_results[[}\StringTok{"980000"}\NormalTok{]]}

\NormalTok{rds\_table }\OtherTok{\textless{}{-}} \FunctionTok{tibble}\NormalTok{(}
  \AttributeTok{Method =} \FunctionTok{c}\NormalTok{(}\StringTok{"RDS{-}I"}\NormalTok{, }\StringTok{"RDS{-}II"}\NormalTok{, }\StringTok{"Successive Sampling"}\NormalTok{),}
  \StringTok{\textasciigrave{}}\AttributeTok{Q36 Estimate}\StringTok{\textasciigrave{}} \OtherTok{=} \FunctionTok{c}\NormalTok{(}
\NormalTok{    rds\_980k}\SpecialCharTok{$}\NormalTok{rds\_i}\SpecialCharTok{$}\NormalTok{q36}\SpecialCharTok{$}\NormalTok{estimate,}
\NormalTok{    rds\_980k}\SpecialCharTok{$}\NormalTok{rds\_ii}\SpecialCharTok{$}\NormalTok{q36}\SpecialCharTok{$}\NormalTok{estimate,}
\NormalTok{    rds\_980k}\SpecialCharTok{$}\NormalTok{rds\_ss}\SpecialCharTok{$}\NormalTok{q36}\SpecialCharTok{$}\NormalTok{estimate}
\NormalTok{  ),}
  \StringTok{\textasciigrave{}}\AttributeTok{Q36 95\% CI}\StringTok{\textasciigrave{}} \OtherTok{=} \FunctionTok{c}\NormalTok{(}
    \FunctionTok{sprintf}\NormalTok{(}\StringTok{"(\%.3f, \%.3f)"}\NormalTok{, rds\_980k}\SpecialCharTok{$}\NormalTok{rds\_i}\SpecialCharTok{$}\NormalTok{q36}\SpecialCharTok{$}\NormalTok{conf.int[}\DecValTok{1}\NormalTok{], rds\_980k}\SpecialCharTok{$}\NormalTok{rds\_i}\SpecialCharTok{$}\NormalTok{q36}\SpecialCharTok{$}\NormalTok{conf.int[}\DecValTok{2}\NormalTok{]),}
    \FunctionTok{sprintf}\NormalTok{(}\StringTok{"(\%.3f, \%.3f)"}\NormalTok{, rds\_980k}\SpecialCharTok{$}\NormalTok{rds\_ii}\SpecialCharTok{$}\NormalTok{q36}\SpecialCharTok{$}\NormalTok{conf.int[}\DecValTok{1}\NormalTok{], rds\_980k}\SpecialCharTok{$}\NormalTok{rds\_ii}\SpecialCharTok{$}\NormalTok{q36}\SpecialCharTok{$}\NormalTok{conf.int[}\DecValTok{2}\NormalTok{]),}
    \FunctionTok{sprintf}\NormalTok{(}\StringTok{"(\%.3f, \%.3f)"}\NormalTok{, rds\_980k}\SpecialCharTok{$}\NormalTok{rds\_ss}\SpecialCharTok{$}\NormalTok{q36}\SpecialCharTok{$}\NormalTok{conf.int[}\DecValTok{1}\NormalTok{], rds\_980k}\SpecialCharTok{$}\NormalTok{rds\_ss}\SpecialCharTok{$}\NormalTok{q36}\SpecialCharTok{$}\NormalTok{conf.int[}\DecValTok{2}\NormalTok{])}
\NormalTok{  ) }\CommentTok{\#...}
\end{Highlighting}
\end{Shaded}

}

\end{table}%

Appendix C: Bootstrap Estimation for Network Scale-Up Using RDS Data

\section{Introduction}\label{introduction-1}

When estimating the size or characteristics of a hidden population using
the Network Scale-Up Method (NSUM), researchers typically assume a
probability sample from the frame population. However, in many applied
settings---including hard-to-reach populations---data are collected via
\textbf{Respondent-Driven Sampling (RDS)}. RDS introduces specific
structural dependencies and inclusion probabilities that violate the
assumptions of simple random sampling.

This presents a challenge: \textbf{how can we correctly estimate
uncertainty for NSUM estimates derived from an RDS sample?} As shown in
\textcite{feeh16-generali} and \textcite{salg06-variance}, the NSUM
estimator depends crucially on inclusion weights \(\pi_i\), which must
reflect the sampling design. When data are RDS-based, these weights are
typically derived from known degree-based estimators such as RDS-II or
Gile's Successive Sampling (SS) weights.

To address this challenge, we propose a \textbf{three-step bootstrap
procedure} for NSUM estimation using RDS data. This approach is
flexible, modular, and applicable across several classes of NSUM
estimators. It separates the issues of: 1. How to resample an RDS chain
(Step 1), 2. How to recalculate sample-specific weights (Step 2), and 3.
How to apply a chosen NSUM estimator (Step 3).

\begin{center}\rule{0.5\linewidth}{0.5pt}\end{center}

\section{Step 1: Resampling the RDS
Sample}\label{step-1-resampling-the-rds-sample}

We begin by resampling from the observed RDS sample in a way that mimics
the original recruitment structure. Let the original sample be: \[
\mathcal{S} = \{i_1, i_2, \dots, i_n\}
\] with recruitment chains and wave indicators. Let \(d_i\) denote
self-reported degree for respondent \(i\), and let the recruitment tree
structure be encoded via seed/recruiter IDs.

\subsection{Options for RDS
Resampling}\label{options-for-rds-resampling}

\begin{itemize}
\tightlist
\item
  \textbf{Tree Bootstrap}: Sample entire recruitment trees (originating
  from seeds) with replacement. This respects the hierarchical
  recruitment structure and allows design effect estimation
  \autocite{salg06-variance}.
\item
  \textbf{Successive Sampling Bootstrap (SSB)}: Sample with replacement
  according to inclusion probabilities derived from the SS model
  (\textcite{gile11-improv}).
\item
  \textbf{Neighborhood Bootstrap}: Use ego-network topology to preserve
  recruitment ties and neighborhood structure
  (\textcite{yauc22-neighboor}).
\end{itemize}

Let \(\mathcal{S}^{(b)}\) denote the sample drawn in bootstrap replicate
\(b\).

\begin{center}\rule{0.5\linewidth}{0.5pt}\end{center}

\section{Step 2: Recalculating
Weights}\label{step-2-recalculating-weights}

NSUM estimators require inclusion weights \(\pi_i\) or their inverses
\(w_i = 1 / \pi_i\). Because bootstrap samples differ in composition and
recruitment pattern, these weights must be \textbf{recomputed for each
replicate}.

\subsection{General Structure}\label{general-structure}

For each replicate \(b\), construct: - \(\mathcal{S}^{(b)}\): resampled
respondent IDs - \(d_i^{(b)}\): degree reports in replicate -
\(\pi_i^{(b)}\): estimated inclusion probabilities

\subsection{Weighting Options}\label{weighting-options}

\begin{itemize}
\item
  \textbf{RDS-II Weights} (\textcite{volz08-simple}): \[
  w_i^{(b)} \propto \frac{1}{d_i^{(b)}}
  \]
\item
  \textbf{SS Weights} (\textcite{gile11-improv}): Incorporate sampling
  fraction and frame size \(N_F\). Computed numerically via successive
  sampling approximation.
\end{itemize}

Let \(\mathbf{X}_i\) denote covariates (e.g.~traits, degree, indicator
of hidden population membership), which are retained from the original
data and passed to Step 3.

\begin{center}\rule{0.5\linewidth}{0.5pt}\end{center}

\section{Step 3: NSUM Estimation}\label{step-3-nsum-estimation}

This step applies an NSUM estimator to the bootstrap sample
\(\mathcal{S}^{(b)}\) using the recalculated weights \(w_i^{(b)}\) and
responses \(y_{i,H}\), where \(y_{i,H}\) is the number of known contacts
respondent \(i\) has in hidden population \(H\).

Let \(N_F\) denote the frame population size (assumed known), and let
\(d_i\) be the degree of respondent \(i\).

\subsection{Generalized NSUM Estimator
(GNSUM)}\label{generalized-nsum-estimator-gnsum}

The weighted GNSUM estimator is:

\[
\hat{N}_H^{(b)} = \frac{\sum_{i \in \mathcal{S}^{(b)}} \frac{y_{i,H}}{\pi_i^{(b)}}}{\sum_{i \in \mathcal{S}^{(b)}} \frac{d_i}{\pi_i^{(b)}}} \cdot N_F
\]

This estimator assumes proportional mixing and equal visibility.

\begin{center}\rule{0.5\linewidth}{0.5pt}\end{center}

\subsection{Symmetric Visibility
Variant}\label{symmetric-visibility-variant}

In our setting, some RDS respondents can be identified \emph{ex post} as
members of the hidden population \(H\). Denote this set
\(\mathcal{H} \subseteq \mathcal{S}\). Under the assumption of
\textbf{symmetric visibility}, we define:

\[
\hat{v}_{H} = \frac{1}{|\mathcal{H}|} \sum_{j \in \mathcal{H}} \frac{y_{j,F}}{d_j}
\]

That is, the average proportion of alters known by members of \(H\) who
are in the frame population \(F\). Incorporating this, the symmetric
visibility GNSUM becomes:

\begin{center}\rule{0.5\linewidth}{0.5pt}\end{center}

\section{Version 2}\label{version-2}

Bootstrap-Based Uncertainty Estimation for NSUM with RDS Samples

\section{1. Introduction}\label{introduction-2}

The \textbf{Network Scale-Up Method (NSUM)} is a powerful tool for
estimating the size of hidden populations. By asking respondents about
the number of people they know who belong to a hidden group, and
calibrating by their social network size, we can estimate the total size
of that group within a known frame population.

However, NSUM typically assumes a \textbf{simple random sample} of the
frame population. In practice, researchers often rely on
\textbf{Respondent-Driven Sampling (RDS)} to access hard-to-reach
populations. RDS is a non-probability sampling method based on peer
referral chains, and it introduces significant complexity due to:

\begin{itemize}
\item
  Unknown inclusion probabilities,
\item
  Dependencies in the recruitment process,
\item
  Homophily on hidden traits,
\item
  Non-uniform degree distributions.
\end{itemize}

To adapt NSUM for use with RDS data, we must \textbf{adjust for the
non-uniform sampling process}. This requires estimating each
respondent's \textbf{inclusion probability} \(\pi_i\), as outlined in
\textcite{feeh16-generaling} and earlier in \textcite{salf06-variance}.
Moreover, because these inclusion probabilities vary across bootstrap
samples, we must \textbf{recompute weights} for each resample.

We propose a three-step bootstrap method for estimating uncertainty in
NSUM from RDS samples:

Step 1: Resampling the RDS Sample Step 2: Recalculate Inclusion
Probabilities Step 3: NSUM Estimation on Resampled and Reweighted Data

This approach is modular, flexible, and compatible with multiple
estimators. At each step, the researcher has multiple decisions to make
(e.g.~\emph{how} to resample, which RDS weights to use, which version of
NSUM estimator to use).

\begin{center}\rule{0.5\linewidth}{0.5pt}\end{center}

\section{2. The Three-Step Bootstrap
Procedure}\label{the-three-step-bootstrap-procedure}

\subsection{Step 1: Resampling the RDS
Sample}\label{step-1-resampling-the-rds-sample-1}

RDS is network-dependent and violates IID assumptions. Therefore,
resampling must preserve recruitment structure, seed variation, and
referral chains.

We consider four strategies:

\begin{enumerate}
\def\labelenumi{\arabic{enumi}.}
\item
  \textbf{Tree Bootstrap}\\
  Entire recruitment trees rooted in seeds are resampled with
  replacement. Each tree \(T_j\) consists of all respondents traced to
  seed \(j\). This preserves hierarchical dependencies and captures
  between-tree heterogeneity (\textcite{salg06-variance}).
\item
  \textbf{Successive Sampling Bootstrap}\\
  Mimics the RDS process as a form of successive sampling from a finite
  population. Each respondent is selected without replacement, with
  probabilities proportional to degree \(d_i\). Requires an assumed
  frame population size \(N\). Implemented in
  \texttt{RDS::gile.ss.weights()} (\textcite{gile11-inference}).
\item
  \textbf{Neighborhood Bootstrap}\\
  Introduced by \textcite{yauc22-neighboot}, this method resamples by
  selecting respondents and replacing them with their \textbf{neighbors}
  in the recruitment graph. This maintains local network dependencies
  and simulates resampling from the underlying contact network.
\item
  \textbf{Chain Bootstrap}\\
  Each bootstrap replicate samples chains or subchains with replacement,
  preserving recruiter--recruitee links. This is implemented in
  \texttt{surveybootstrap::rds.boot.draw.chain()} and used in studies
  like \textcite{weir12-comparison}.
\end{enumerate}

Each resample produces a new dataset \(\mathcal{S}^{(b)}\), which is
passed to Step 2.

\begin{center}\rule{0.5\linewidth}{0.5pt}\end{center}

\subsection{Step 2: Recalculate Inclusion
Probabilities}\label{step-2-recalculate-inclusion-probabilities}

RDS produces samples with \textbf{unknown and unequal probabilities of
inclusion}, which must be corrected when used in NSUM estimation. This
is achieved by estimating the probability \(\pi_i^{(b)}\) that each
individual \(i\) in bootstrap replicate \(b\) is included in the sample,
conditional on their degree and position in the recruitment tree.

These probabilities are used to generate \textbf{sampling weights}
\(w_i^{(b)} = 1/\pi_i^{(b)}\), which are passed into NSUM estimators.

\subsubsection{Estimation Methods}\label{estimation-methods}

\textbf{(a) RDS-II (Volz-Heckathorn) Weights}\\
Assumes the probability of selection is proportional to the respondent's
degree:

\[\pi_i \propto d_i \quad \Rightarrow \quad w_i = \frac{1}{d_i}\]

These weights are normalized post hoc. This method is simple but does
not account for homophily or finite population correction
(\textcite{volz08-rds}).

\textbf{(b) Gile's Successive Sampling (SS) Weights}\\
This method assumes RDS approximates successive sampling without
replacement. The inclusion probabilities are computed by simulating from
a known or assumed population size \(N\). This method is implemented in
\texttt{RDS::gile.ss.weights()} and adjusts for:

\begin{itemize}
\item
  Finite population effects,
\item
  Degree-based sampling,
\item
  Sample depletion over waves.
\end{itemize}

\textbf{(c) User-Defined or Model-Based Weights}\\
Researchers may define weights using alternative models or Bayesian
simulations. This includes post-stratification or fitting full
generative models of the RDS process.

\subsubsection{Software Example (R)}\label{software-example-r}

\begin{Shaded}
\begin{Highlighting}[]
\FunctionTok{library}\NormalTok{(RDS)}
\NormalTok{rd }\OtherTok{\textless{}{-}} \FunctionTok{as.rds.data.frame}\NormalTok{(boot\_sample, }\AttributeTok{id =} \StringTok{"id"}\NormalTok{, }\AttributeTok{recruiter.id =} \StringTok{"recruiter.id"}\NormalTok{)}
\NormalTok{boot\_sample}\SpecialCharTok{$}\NormalTok{ss\_weight }\OtherTok{\textless{}{-}} \FunctionTok{gile.ss.weights}\NormalTok{(rd, }\AttributeTok{N =} \DecValTok{980000}\NormalTok{)}
\NormalTok{boot\_sample}\SpecialCharTok{$}\NormalTok{vh\_weight }\OtherTok{\textless{}{-}} \FunctionTok{rds.weights}\NormalTok{(rd, }\AttributeTok{weight.type =} \StringTok{"RDS{-}II"}\NormalTok{)}
\end{Highlighting}
\end{Shaded}

The output is a new dataset with respondent traits, degrees, and updated
\(\pi_i^{(b)}\), which are used in Step 3.

\begin{center}\rule{0.5\linewidth}{0.5pt}\end{center}

\subsection{Step 3: NSUM Estimation on Resampled and Reweighted
Data}\label{step-3-nsum-estimation-on-resampled-and-reweighted-data}

Given a bootstrap sample \(\mathcal{S}^{(b)}\), we estimate the size of
the hidden population \(N_H\) using one of several NSUM estimators. All
estimators rely on weighted sums of out-reports and degree.

\subsubsection{(a) Generalized NSUM
(GNSUM)}\label{a-generalized-nsum-gnsum}

As described in \textcite{feeh16-generaling}, GNSUM estimates:

\[\hat{N}_H^{(b)} = \left( \frac{\sum_{i} \frac{y_{i,H}}{\pi_i^{(b)}}}{\sum_{i} \frac{d_i}{\pi_i^{(b)}}} \right) \cdot N_F\]

Where:

\begin{itemize}
\item
  \(y_{i,H}\): number of known alters in the hidden population,
\item
  \(d_i\): degree (known network size, e.g.~from Q13),
\item
  \(N_F\): size of the frame population (e.g.~domestic workers in UK).
\end{itemize}

\subsubsection{(b) GNSUM with Symmetric Visibility (for Hidden Members
in
RDS)}\label{b-gnsum-with-symmetric-visibility-for-hidden-members-in-rds}

In your context, some RDS respondents are ex post identified as members
of the hidden population. Under the \textbf{symmetric visibility
assumption}, if \(i \in H\), we assume the probability that \(i\) knows
\(j\) is the same as \(j\) knows \(i\). This allows \textbf{in-reports}
to be incorporated.

Let:

\begin{itemize}
\item
  \(I_H(i) = 1\) if \(i \in H\), 0 otherwise
\item
  \(y_{i}^{\text{in}}\): number of other hidden members who report
  knowing \(i\)
\end{itemize}

Then the estimator becomes:

\[\hat{N}_H^{(b)} = \left( \frac{\sum_{i} \frac{y_{i,H} + I_H(i) \cdot y_{i}^{\text{in}}}{\pi_i^{(b)}}}{\sum_{i} \frac{d_i}{\pi_i^{(b)}}} \right) \cdot N_F\]

\subsubsection{(c) Modified Basic Scale-Up
(MBSU)}\label{c-modified-basic-scale-up-mbsu}

This estimator adjusts for key reporting biases via three correction
factors:

\[\hat{N}_H^{(b)} = \left( \frac{\sum_{i} \frac{y_{i,H}}{\pi_i^{(b)}}}{\sum_{i} \frac{d_i}{\pi_i^{(b)}}} \right) \cdot N_F \cdot \frac{1}{\delta \cdot \tau \cdot \rho}\]

Where:

\begin{itemize}
\item
  \(\delta\): \textbf{Transmission bias} (probability respondent is
  aware of alter's status),
\item
  \(\tau\): \textbf{Barrier effect} (social mixing between hidden and
  frame population),
\item
  \(\rho\): \textbf{Popularity bias} (relative visibility of hidden
  population members).
\end{itemize}

These values can be:

\begin{itemize}
\item
  Estimated using \textbf{known subpopulations} (e.g., alter groups with
  known size),
\item
  Set by expert \textbf{elicitation},
\item
  Scanned in \textbf{sensitivity analysis}.
\end{itemize}

\subsubsection{(d) Model-Based NSUM}\label{d-model-based-nsum}

Bayesian implementations (e.g. \textcite{malt15-estimating}) model:

\begin{itemize}
\item
  Degree distributions,
\item
  Reporting error,
\item
  Group visibility,
\item
  Hidden population size.
\end{itemize}

They yield a \textbf{posterior distribution} over \(N_H\), and
uncertainty is embedded within the model.

Software:

\begin{itemize}
\item
  \texttt{NSUMBayes} (in \texttt{R})
\item
  Custom MCMC in \texttt{stan} or \texttt{JAGS}
\end{itemize}

\begin{center}\rule{0.5\linewidth}{0.5pt}\end{center}

\subsection{4. Aggregating Bootstrap
Estimates}\label{aggregating-bootstrap-estimates}

After \(B\) replicates:

\[\hat{N}_H^{(1)}, \dots, \hat{N}_H^{(B)}\]

We compute:

\begin{itemize}
\item
  \textbf{Point estimate}:
  \(\bar{N}_H = \frac{1}{B} \sum_b \hat{N}_H^{(b)}\)
\item
  \textbf{Standard error}: sample SD
\item
  \textbf{95\% CI}: empirical percentiles (e.g., 2.5\%, 97.5\%)
\end{itemize}

\begin{center}\rule{0.5\linewidth}{0.5pt}\end{center}

\subsection{5. References}\label{references-1}

\begin{itemize}
\item
  \textcite{feeh16-generaling}
\item
  \textcite{salf06-variance}
\item
  \textcite{gile11-inference}
\item
  \textcite{volz08-rds}
\item
  \textcite{malt15-estimating}
\item
  \textcite{yauc22-neighboot}
\item
  \textcite{weir12-comparison}
\item
  \textcite{salg06-variance}
\item
  \textcite{rust96-rescaled}
\item
  \textcite{rao88-resampling}
\end{itemize}





\end{document}
