% Options for packages loaded elsewhere
% Options for packages loaded elsewhere
\PassOptionsToPackage{unicode}{hyperref}
\PassOptionsToPackage{hyphens}{url}
\PassOptionsToPackage{dvipsnames,svgnames,x11names}{xcolor}
%
\documentclass[
  12pt,
  letterpaper,
  DIV=11,
  numbers=noendperiod]{scrartcl}
\usepackage{xcolor}
\usepackage[margin=1in]{geometry}
\usepackage{amsmath,amssymb}
\setcounter{secnumdepth}{5}
\usepackage{iftex}
\ifPDFTeX
  \usepackage[T1]{fontenc}
  \usepackage[utf8]{inputenc}
  \usepackage{textcomp} % provide euro and other symbols
\else % if luatex or xetex
  \usepackage{unicode-math} % this also loads fontspec
  \defaultfontfeatures{Scale=MatchLowercase}
  \defaultfontfeatures[\rmfamily]{Ligatures=TeX,Scale=1}
\fi
\usepackage{lmodern}
\ifPDFTeX\else
  % xetex/luatex font selection
\fi
% Use upquote if available, for straight quotes in verbatim environments
\IfFileExists{upquote.sty}{\usepackage{upquote}}{}
\IfFileExists{microtype.sty}{% use microtype if available
  \usepackage[]{microtype}
  \UseMicrotypeSet[protrusion]{basicmath} % disable protrusion for tt fonts
}{}
\makeatletter
\@ifundefined{KOMAClassName}{% if non-KOMA class
  \IfFileExists{parskip.sty}{%
    \usepackage{parskip}
  }{% else
    \setlength{\parindent}{0pt}
    \setlength{\parskip}{6pt plus 2pt minus 1pt}}
}{% if KOMA class
  \KOMAoptions{parskip=half}}
\makeatother
% Make \paragraph and \subparagraph free-standing
\makeatletter
\ifx\paragraph\undefined\else
  \let\oldparagraph\paragraph
  \renewcommand{\paragraph}{
    \@ifstar
      \xxxParagraphStar
      \xxxParagraphNoStar
  }
  \newcommand{\xxxParagraphStar}[1]{\oldparagraph*{#1}\mbox{}}
  \newcommand{\xxxParagraphNoStar}[1]{\oldparagraph{#1}\mbox{}}
\fi
\ifx\subparagraph\undefined\else
  \let\oldsubparagraph\subparagraph
  \renewcommand{\subparagraph}{
    \@ifstar
      \xxxSubParagraphStar
      \xxxSubParagraphNoStar
  }
  \newcommand{\xxxSubParagraphStar}[1]{\oldsubparagraph*{#1}\mbox{}}
  \newcommand{\xxxSubParagraphNoStar}[1]{\oldsubparagraph{#1}\mbox{}}
\fi
\makeatother


\usepackage{longtable,booktabs,array}
\usepackage{calc} % for calculating minipage widths
% Correct order of tables after \paragraph or \subparagraph
\usepackage{etoolbox}
\makeatletter
\patchcmd\longtable{\par}{\if@noskipsec\mbox{}\fi\par}{}{}
\makeatother
% Allow footnotes in longtable head/foot
\IfFileExists{footnotehyper.sty}{\usepackage{footnotehyper}}{\usepackage{footnote}}
\makesavenoteenv{longtable}
\usepackage{graphicx}
\makeatletter
\newsavebox\pandoc@box
\newcommand*\pandocbounded[1]{% scales image to fit in text height/width
  \sbox\pandoc@box{#1}%
  \Gscale@div\@tempa{\textheight}{\dimexpr\ht\pandoc@box+\dp\pandoc@box\relax}%
  \Gscale@div\@tempb{\linewidth}{\wd\pandoc@box}%
  \ifdim\@tempb\p@<\@tempa\p@\let\@tempa\@tempb\fi% select the smaller of both
  \ifdim\@tempa\p@<\p@\scalebox{\@tempa}{\usebox\pandoc@box}%
  \else\usebox{\pandoc@box}%
  \fi%
}
% Set default figure placement to htbp
\def\fps@figure{htbp}
\makeatother





\setlength{\emergencystretch}{3em} % prevent overfull lines

\providecommand{\tightlist}{%
  \setlength{\itemsep}{0pt}\setlength{\parskip}{0pt}}



 
\usepackage[backend=biber,natbib =
true,style=apa,sorting=nyt,maxcitenames=2]{biblatex}
\addbibresource{references.bib}
\addbibresource{MyLibrary2025-08-25.bib}


% Do NOT load biblatex here.
\DeclareLanguageMapping{english}{english-apa}
\usepackage{amsthm}
\usepackage{amsmath}
\usepackage{amsfonts}
\usepackage{amssymb}
\usepackage{float}
\usepackage{caption}
\usepackage{subcaption}
\usepackage{stmaryrd}
\usepackage{pdflscape}  % ADD THIS for landscape pages
\theoremstyle{plain}
\newtheorem{theorem}{Theorem}[section]
\newtheorem{proposition}[theorem]{Proposition}
\theoremstyle{definition}
\newtheorem{definition}{Definition}
\newtheorem{corollary}{Corollary}
\newtheorem{example}{Example}
\renewenvironment{proof}
   {\par\noindent\textbf{Proof.}\ }
   {\hfill$\blacksquare$\par}
\usepackage[ruled,vlined,linesnumbered]{algorithm2e}
\KOMAoption{captions}{tableheading}
\makeatletter
\@ifpackageloaded{caption}{}{\usepackage{caption}}
\AtBeginDocument{%
\ifdefined\contentsname
  \renewcommand*\contentsname{Table of contents}
\else
  \newcommand\contentsname{Table of contents}
\fi
\ifdefined\listfigurename
  \renewcommand*\listfigurename{List of Figures}
\else
  \newcommand\listfigurename{List of Figures}
\fi
\ifdefined\listtablename
  \renewcommand*\listtablename{List of Tables}
\else
  \newcommand\listtablename{List of Tables}
\fi
\ifdefined\figurename
  \renewcommand*\figurename{Figure}
\else
  \newcommand\figurename{Figure}
\fi
\ifdefined\tablename
  \renewcommand*\tablename{Table}
\else
  \newcommand\tablename{Table}
\fi
}
\@ifpackageloaded{float}{}{\usepackage{float}}
\floatstyle{ruled}
\@ifundefined{c@chapter}{\newfloat{codelisting}{h}{lop}}{\newfloat{codelisting}{h}{lop}[chapter]}
\floatname{codelisting}{Listing}
\newcommand*\listoflistings{\listof{codelisting}{List of Listings}}
\makeatother
\makeatletter
\makeatother
\makeatletter
\@ifpackageloaded{caption}{}{\usepackage{caption}}
\@ifpackageloaded{subcaption}{}{\usepackage{subcaption}}
\makeatother
\usepackage{bookmark}
\IfFileExists{xurl.sty}{\usepackage{xurl}}{} % add URL line breaks if available
\urlstyle{same}
\hypersetup{
  pdftitle={Quantifying Hidden Exploitation: Dual-Method Prevalence Estimates of Modern Slavery Risk Among UK Domestic Workers},
  pdfauthor={Caroline Emberson; Scott Moser},
  colorlinks=true,
  linkcolor={black},
  filecolor={Maroon},
  citecolor={RoyalBlue},
  urlcolor={BrickRed},
  pdfcreator={LaTeX via pandoc}}


\title{Quantifying Hidden Exploitation: Dual-Method Prevalence Estimates
of Modern Slavery Risk Among UK Domestic Workers\footnote{Authors' names
  are listed in alphabetical order.}}
\author{Caroline Emberson \and Scott Moser}
\date{08, October 2025}
\begin{document}
\maketitle
\begin{abstract}
Purpose

The purpose of this article is to demonstrate a quantitative approach to
the construction of a risk index of labour exploitation and alternative
estimators of the prevalence of exploitation.

Design/ Methodology/ Approach

Using data from a survey of domestic workers based in the United Kingdom
(UK), we use statistical techniques, including Respondent Driven
Sampling (RDS) methods RDS-I and RDS-II and Network Scale Up (NSUM)
methods, to produce an index of labour exploitation risk and estimators
of the prevalence of labour exploitation.

Findings

The labour exploitation risk index shows a reverse correlation between
the increasing seriousness of exploitation and the number of
exploitation cases reported. The various prevalence estimators examined
show significant differences in population level exploitation.

Research implications/ limitations

Further research into the application of different quantitative
statistical estimators of the prevalence of labour exploitation is
urgently required.

Practical implications

Robust estimators are necessary if policy makers are to make informed
choices about the appropriate allocation of scarce resources to help to
eradicate severe forms of labour exploitation and labour abuse.

Social implications

Even by more conservative estimates, thousands of domestic workers in
the UK are subject to labour exploitation. Urgent policy attention is
needed if structural vulnerabilities are to be removed.

Originality

We believe this paper is the first to compare the use of RDS and NSUM
methods in the quantitative estimation of the prevalence of labour
exploitation and to construct a quantitative, composite index of labour
exploitation risk.
\end{abstract}

\renewcommand*\contentsname{Table of contents}
{
\hypersetup{linkcolor=}
\setcounter{tocdepth}{3}
\tableofcontents
}

\newpage

\section{Introduction}\label{introduction}

Labour exploitation has been defined as `work situations that deviate 
significantly from standard working conditions as defined by legislation
or other binding legal regulations, concerning in particular
remuneration, working hours, leave entitlements, health and safety
standards and decent treatment'\autocite[10]{european_union_for_fundamental_rights_severe_2015}. In the operations and supply chain management literature,
interest in businesses' respect for these kinds of employee labour rights began with
studies focused upon labour rights transgressions related to risk
reduction and risk communication and how to improve
employees' health and safety \autocite{chinander_aligning_2001,wolf_operationalizing_2001}. More recently, 
serious labour rights abuses have come to the fore with studies
examining the challenges of severe labour exploitation under the umbrella term `modern slavery' \autocite{gold_modern_2015,new_modern_2015,benstead_horizontal_2018,stevenson_modern_2018}. While this literature
offers important insights into severe forms of labour exploitation, particularly
in global supply chains, this and the wider social sustainability literature has
been criticised for its de-humanised approach to the understanding
of workers and their working conditions \autocite{soundararajan_humanizing_2021}. While at least one current
project seeks to examine the phenomenon of worker voice in factory
settings  \autocite{leverhulme_trust_research_2022}, there appears to be little attention paid to severe forms of labour exploitation from the workers' perspective in the private sphere. Nowhere are the realities of individual workers' experiences of employer exploitation brought into sharper relief than in the setting of domestic
work in private households.

The authors of the Global Slavery Index estimate that there are
seventy-six million people employed in domestic work worldwide
\autocite{international_labour_organization_global_2022}).
According to \textcite{bonnet_domestic_2022}, eighty percent of this domestic work is unregulated and informal. Labour
exploitation has been identified as an extensive global problem within
the sector, with domestic work identified as one of five private sector
groupings which contribute the most to forced labour. Defined in the ILO
Forced Labour Convention, 1930 No.29, forced or compulsory labour is
`all work or service which is exacted from a person under the threat of
a penalty and for which the person has not offered himself or herself
voluntarily' \autocite{ilo_what_2024}. Seventy-six
percent of domestic workers are women, and these workers represent four
percent of the total female workforce \autocite{international_labour_organization_global_2022}. Indeed, women in forced labour are much more likely to be in domestic
work than in any other occupation \autocite{international_labour_organization_global_2022}.
The ILO suggest that female domestic workers may be coerced through
non-payment of wages; abuse of vulnerability; subjected to physical and
sexual violence or experience threats against their family members. Such
severe forms of labour exploitation may be present alongside other,
perhaps less severe but equally illegal, practices which constitute
various forms of labour abuse. The criminalisation of both labour
exploitation and abuse in a domestic setting has developed in recent
times, with legislation enacted in the United Kingdom (UK), Europe,
Australia and Norway to criminalise such severe exploitation under the
term `modern slavery'. However, even where modern slavery laws are in
place, reliance on traditional, inspection-led, approaches to detection
designed primarily to ensure labour rights compliance within communal
workplaces such as factories mean that the number of reported cases of
labour exploitation in private dwellings may well severely underestimate
actual exploitation levels. Though an exploration of labour exploitation within private residences, our research seeks to redress the paucity of rigorous quantitative research in the modern slavery field.  Specifically, this article aims to contribute to a more
nuanced understanding of how quantitative methodologies may be deployed to
improve understanding of the realities of workers' conditions by
demonstrating the use of a statistically robust estimation of the nature
and proportion of labour exploitation and abuse among domestic workers in the UK. This setting was chosen due to long-standing national
legislation criminalising modern slavery introduced to the UK in 2015.
Despite, or perhaps because of this legislation, in recent years the
number of potential victims entering the UK's National Referral
Mechanism (NRM), a scheme which provides government support for those
suspected to be modern slavery survivors, has continued to increase.
Nineteen thousand, one hundred and twenty-five potential victims were
recorded in 2024: the highest annual figure since the NRM began
\autocite{home_office_modern_2025}. In 2024, for the
first time the number of cases of potential modern slavery among females
handled by the charity Unseen, who run the UK's modern slavery helpline,
were more prevalent than those among men 
(\cite{carter_women_2025}). Despite these worrying
headline statistics, and the persistence of specific concerns about high
levels of exploitation among domestic workers in the grey literature
\autocite{kalayaan_new_2008, mantouvalou_modern_2016, latin_american_womens_rights_service_behind_2023}, to our knowledge no-one has yet estimated the nature and extent of labour exploitation and abuse that may exist among domestic workers in the UK.

In contrast to overseas factory workers in globally dispersed, product,
supply chains, many service workers engaged in domestic work have
migrated to work in the UK. These transnational workers enter on
restricted visas where their employment---and their right to remain in
the country---is tied to their continuing employment. It is now ten
years since the UK's Modern Slavery Act was enacted. During its passage
through parliament, those advocating for the rights of domestic workers
were successful in expanding the final category boundaries of the
legislation to include, in Section 53, the specific definition of
(overseas) domestic workers as modern slavery victims 
\autocite{caruana_boundaries_2025}. These transnational
migrants are at particular risk of exploitation due to regulatory visa
restrictions and intersecting structural issues related to their gender,
the relative isolation of domestic work and a lack of supportive social
networks. This can mean that they fall out of legal migratory status.
Due to the social stigma attached to such illegal working, transnational
workers remaining in the UK without the right to work may be considered
a hidden, hard-to-reach, population. Extracting a sample of domestic
workers which includes this group raises difficulties when trying to
employ the normal statistical sampling methods considered necessary for
robust prevalence estimation. Perhaps due to these sampling
difficulties, we know relatively little about the nature of labour
exploitation among this particularly at-risk group of workers.
Fortunately, there has been significant interest in the development of
alternative methods for prevalence estimation which include such
hard-to-reach groups, with many scholars advocating and developing the
use of respondent-driven sampling (RDS) techniques to support
statistically robust estimators.

In this paper, we make two specific contributions to the operations and
supply chain management literature. First, we demonstrate the use of RDS
coupled with Network Scale-up Methods (N-SUM) to reach and sample
respondents' views of their working conditions among these,
predominantly female, transnational migrant domestic workers. We use the
data we obtain from these respondents to show how quantitative survey data can be
used to estimate the proportion of workers experiencing labour
exploitation. Second, we begin to capture the nature and extent of
modern slavery as voiced by domestic workers, thus, we
believe, expanding the nascent literature on worker voice which has, in
the main, focused primarily on factory workers
\autocite{stephens_theorising_2024}. These contributions
not only extend our understanding of the risks of labour exploitation
and abuse among service workers engaged in domestic settings but also
show how it is possible to shed light on the severity of the
individuals' experience of exploitation through the construction of a
novel risk index. The remainder of this paper is structured as follows.
our study in more detail, highlighting what is already known about the
current population of domestic workers in the UK and the conditions in
which they work. Next, we describe our research methods. We review the
development of the respondent-driven sampling (RDS) techniques we used
and explain why this sampling method is suitable for our study. We then
describe our survey methods, including how we designed our survey
instrument, contacted our sample seeds and analysed our data. We then
present and discuss our findings, detailing the proportional estimate
that we calculated and the risk index we constructed. In our discussion,
we expand upon the implications of our findings for government policy,
enforcement practices and further research, including how these methods
may be used in future studies of labour exploitation in other sectoral
and geographic contexts. The limitations of our study are outlined,
before, finally, we conclude our article.

\subsection{Conceptualising Labour Exploitation and the Degree of
Risk}\label{conceptualising-labour-exploitation-and-the-degree-of-risk}

\begin{itemize}
\item
  Binary vs.~continuous definitions.
\item
  Risk index construction and theoretical justification.
\end{itemize}

Modern slavery has been criticised by some for its overly extensive
scope: encapsulating a broad range of divergent sub-categories of
exploitation (\textcite{oconnell_davidson_margins_2015};
\textcite{gutierrez-huerter_o_change_2023}). For this reason, we used
the International Labour Organization's (\textcite{ILO11-indicators})
`Indicators of Forced Labour' to identify the potential for severe
labour exploitation and as a basis for the quantification of our labour
exploitation and abuse risk index. The ILO identify eleven indicators
designed to help understand how forced labour arises and how it affects
victims. These indicators include: abuse of vulnerability; deception;
restriction of movement; isolation; physical and sexual violence;
intimidation and threats; retention of identity documents; withholding
of wages; debt bondage; abusive working and living conditions and
excessive overtime. According to the ILO, the presence of a single
indicator in any given situation may in some cases imply the existence
of forced labour. However, it also suggests that in other cases it may
be necessary to look for several indications which, taken together, may
point to a case of forced labour. We seek to refine this statement
through the construction of a composite index by which means a degree of
risk related to the likelihood of a domestic worker experiencing this
most severe form of exploitation may be distinguished from the likely
occurrence of less severe, though similarly illegal, forms of labour
abuse.

\section{Evaluating the Degree of
Risk}\label{evaluating-the-degree-of-risk}

The study of risk management has a long tradition in operations and
supply chain management. Initially, the risks under consideration were
primarily related to ensuring continuity of the supply of goods and
services (see for example, \textcite{juttner_supply_2003}). Beginning
with \textcite{anderson_critical_2006} and
\textcite{anderson_sustainability_2009}, however, a literature stream of
sustainability-related supply chain risk management developed related
specifically to the risks associated with the environment and social
justice. A normative consensus related to the main stages of supply
chain risk management has developed in the literature, with a five-stage
sequential model typically presented. There have also been empirical
studies of risk management within various industrial supply chains in
the United States and India (\textcite{tarei_hybrid_2018};
\textcite{dellana_scale_2021}), including the quantification of a risk
index for the petroleum supply chain (\textcite{tarei_hybrid_2018}).
Yet, while these authors recognize the need for responsible management
and its effect on societal values, in line with other literature in the
field they view risk from the perspective of the corporate supply chain
rather than examining the risk of harm to the worker.

In our study, we conceptualise the risk of labour exploitation from the
workers' perspective. We conceive severe forms of labour exploitation
such as forced labour as one end of a spectrum ranging from illegal
employment practices that constitute labour abuse, such as wage payments
below legal minimum wage levels and health and safety violations,
through to the likelihood of criminal exploitation recognized in the UK
as modern slavery. Our assessment of this personal risk permits a degree
of risk to be assigned to various clusters of forced labour indicators
with the more indicators present, the stronger the likelihood that the
working conditions may be considered exploitative. Our approach,
therefore, includes, but goes beyond, assessing the likelihood of forced
labour by simply quantifying the proportion of survivors entering the
UK's National Referral Mechanism (NRM): a government system for survivor
support set up to identify whether there are positive grounds for the
identification of Modern Slavery. In our method, an NRM referral is used
as the strongest indicator of modern slavery risk, with lesser risks
assessed according to the degree to which cumulative indicators of
forced labour are reported.

\subsection{Case Setting: Labour Exploitation Risk Among Transnational
Migrant Domestic Workers In The
UK}\label{case-setting-labour-exploitation-risk-among-transnational-migrant-domestic-workers-in-the-uk}

Domestic work forms part of a broader industrial category of Personal
and Household Service work (PHS). Work in this category includes those
employed in `social work activities without accommodation' and
`activities of households as employers of domestic personnel'
(\textcite{european_commission_staff_2012}). In 2017, an estimated
980,000 people were engaged in PHS work in the UK
(\textcite{manoudi_analysis_2018}). \textcite{manoudi_analysis_2018}
highlight that the PHS sector is dominated by women and migrants, with
many undeclared foreign workers. Detailed statistics related to the
country of origin of domestic workers migrating to work in PHS in the UK
are difficult to isolate before 2019. Since that time, annual migration
has fluctuated -- falling sharply in 2021 due in part to the COVID-19
pandemic, before later rising again above pre-pandemic levels. In the
year to December 2022, the UK Home Office reported that it had issued
18,533 Overseas Domestic Worker visas (\textcite{home_office_why_2023}).
These domestic workers came from various countries in South America and
Asia, including many from the Philippines.

In 2023, \textcite{strauss_britain_2023} reported a big shift in the
source countries of migrants arriving in the UK on the Overseas Domestic
Worker and other types of worker visas. Transnational domestic workers
from the Philippines and India accounted for the single largest number
of applications granted (10,186 and 3,858 visas respectively), followed
by smaller, but still significant, numbers of workers arriving from
Bangladesh (465), Nigeria (446), Sri Lanka (444), Egypt (422), and
Ethiopia (285). In the same period, smaller numbers of visa applications
to work as a domestic worker in the UK were also accepted from workers
from other source countries including, but not limited to, the Sudan,
Nepal, Ghana, Kenya, Lebanon, Eritrea, Iran, Turkey, Yemen, Malaysia,
Thailand, and Morocco. This post-Brexit increase in the diversity of
source countries from which transnational workers are drawn makes a more
detailed analysis of the risk of labour exploitation in the sector both
timely and more urgent.

There is a long history of reports of exploitation in the domestic work
sector in the UK. In 2008, the civil society organisation Kalayaan,
which was formed to campaign for the formal recognition of migrant
domestic workers' rights in the UK, reported on the impact of proposed
changes to the UK immigration system on migrant domestic workers
(\textcite{kalayaan_new_2008}). Their report highlights government
recognition of documented and unacceptable levels of abuse and
exploitation among domestic workers in the UK as early as 1996. At this
stage, new policies, including the development of a specialised visa
allowing domestic workers to change employer during their stay were
introduced. However, in 2012, these visa conditions were modified, tying
domestic workers to a single employer and restricting the length of time
that they are permitted to remain in the country to a period of six
months (\textcite{gower_calls_2016}). Overseas domestic worker visa
holders are now, again, permitted to change employers, but not to apply
to renew their six-month long visa unless they receive a positive
`Conclusive Grounds' decision related to exploitation considered to be
modern slavery through the UK's National Referral Mechanism (NRM)
(\textcite{romero_blueprint_2025}).

These reports highlight the underlying reasons for migrant domestic
workers' vulnerability, including workers' relative desperation for
work; their lack of social ties; unfamiliarity with English language and
culture; long working hours; lack of knowledge of their legal rights; a
lack of oversight of the private home as a workplace; their work forming
part of the informal economy; their reliance on their employer for
permission to work in the UK; and their lack of recourse to public
funds. As a result, migrant domestic workers are vulnerable to abuse
ranging from minor breaches of employment and health and safety law, to
physical and sexual violence, slavery, forced labour and trafficking.

That these conditions may persist is evidenced by a report from another
civil society organisation, the Latin American Women's Rights Service,
which describes the results from twelve in-depth interviews with Latin
American domestic workers in the UK. This report depicts high levels of
isolation, exploitation and abuse including a failure by employers to
provide written contracts or payslips; breaches of verbal agreements; a
requirement to perform different tasks from those indicated during
recruitment; increasing working hours with little or no time off;
excessive work days; a lack of paid holiday; many domestic workers not
registered with a GP; sexual harassment in the workplace; verbal or
physical abuse; employer surveillance; a lack of opportunity to change
working conditions; isolation and fear of seeking help; and high
reported levels of trafficking for labour exploitation
(\textcite{latin_american_womens_rights_service_behind_2023}).

Against this backdrop, we used respondent driven sampling (RDS) as a
sampling technique to recruit and survey domestic workers in the UK
about the working conditions they were experiencing to estimate the
nature and scale of abuse and exploitation based upon reports of their
conditions by domestic workers themselves.

\section{Methods and Estimation
Strategy}\label{methods-and-estimation-strategy}

\subsection{Study Design and Sampling}\label{study-design-and-sampling}

Data for this study were collected through a bespoke respondent-driven
sampling (RDS) survey conducted among domestic workers in the United
Kingdom. RDS was selected as the most appropriate sampling strategy
because it is specifically designed for reaching hidden or
hard-to-access populations where no comprehensive sampling frame exists.
The approach builds on peer-to-peer recruitment through existing social
networks and allows the derivation of statistically valid population
estimates once appropriate adjustments are made for network size and
recruitment patterns.

The survey began with a small set of initial ``seed'' participants.
These seeds were identified through community organisations and support
networks that regularly interact with domestic workers. Care was taken
to ensure that the seeds reflected variation across the three principal
subgroups of the study---Latinx, Filipino, and British workers---as well
as diversity in gender, employment arrangements (live-in and live-out),
and geographical location within the UK. Each seed received a limited
number of uniquely coded recruitment coupons and was instructed to
recruit peers who also met the eligibility criteria for the study.
Recruitment proceeded in successive waves, as each new participant was
given coupons to recruit additional respondents from within their own
networks.

To encourage participation, a dual-incentive structure was implemented.
Participants received a small cash or electronic payment upon completion
of the survey and an additional incentive for each eligible person they
successfully recruited into the study. Incentive amounts were set in
consultation with partner organisations to ensure they were modest yet
sufficient to motivate participation without coercion.

Data integrity and verification were central to the design. Each survey
record was associated with a unique coupon code that linked recruiters
and recruits, enabling tracking of referral chains. Duplicate entries
were prevented through real-time monitoring of coupon codes and checks
of contact details. In addition, data patterns were examined during
fieldwork to identify irregular recruitment activity or potential
fabrication. Verification protocols were informed by best practice
guidelines in RDS implementation and ensured that the final dataset
consisted of unique, valid respondents.

Recruitment continued until equilibrium was reached on key demographic
and employment variables, indicating that the sample composition had
stabilised across successive waves. In total, the study achieved a final
analytic sample of \emph{N = XX} respondents distributed across \emph{X}
recruitment waves. The length and structure of the recruitment chains
varied by subgroup, with Latinx respondents typically forming longer and
denser referral trees, while British respondents tended to have shorter
recruitment paths. This pattern reflects known differences in the
density and cohesion of social networks within the domestic work sector.

Overall, the RDS approach provided a practical and statistically
defensible means of accessing a population for which no comprehensive
administrative or membership lists exist. The resulting sample formed
the basis for subsequent estimation procedures, including both RDS-based
and network scale-up (NSUM) estimations.

\subsection{Survey Instrument and
Indicators}\label{survey-instrument-and-indicators}

The survey instrument was designed specifically for this study to
capture both the characteristics of domestic workers in the United
Kingdom and the structure of their social networks. The questionnaire
was developed in consultation with sector organisations, including
migrant worker support groups and labour rights charities, to ensure
that it was culturally appropriate and reflected the realities of
domestic work in the UK. The survey was administered in English,
Spanish, and Tagalog, with bilingual facilitators available to assist
respondents when necessary.

The survey consisted of four core modules: (1) demographic and
employment characteristics, (2) social network size and composition, (3)
indicators of labour exploitation, and (4) knowledge of employment
rights and working conditions. The instrument was structured to allow
both \textbf{ego-based} and \textbf{alter-based} information to be
collected, which later enabled the use of both respondent-driven
sampling (RDS) and network scale-up (NSUM) estimation methods.

The \textbf{demographic and employment} module included items on age,
gender, country of origin, migration history, visa status (where
relevant), length of residence in the UK, type of domestic work
performed, hours worked, and employment arrangements (live-in or
live-out). These data were used to describe the sample and to explore
subgroup differences across Latinx, Filipino, and British domestic
workers.

The \textbf{social network} module gathered ego-based information
necessary for RDS estimation. Respondents were asked to report the
number of other domestic workers they personally knew in the UK who also
owned a mobile phone and with whom they had been in contact during the
past month. This network size variable was used to estimate each
respondent's degree, which forms a key input for RDS weighting
procedures. Respondents were also asked to report information about who
recruited them into the study, allowing construction of complete
recruitment trees and verification of recruitment wave progression.

The \textbf{alter-based network questions} were designed to capture
information suitable for NSUM estimation. Respondents were asked
questions of the form, ``How many domestic workers do you know who have
experienced any of the following situations?'' followed by a list of
specific indicators of exploitation. These questions provided estimates
of the proportion of alters within each respondent's network who met
defined criteria related to exploitation, thereby enabling inference to
the broader population of domestic workers.

Indicators of labour exploitation were adapted from the International
Labour Organization's (ILO) framework of forced labour and included
domains such as excessive working hours, non-payment or delayed payment
of wages, restriction of movement, confiscation of documents, and
threats or intimidation by employers. These indicators were used in two
distinct ways.

First, they were used to construct a \textbf{binary classification} of
exploitation. Respondents were classified as exploited if they reported
experiencing one or more of the defined ILO indicators above a specified
threshold. This binary outcome allowed for prevalence estimation using
both RDS and NSUM methods.

Second, the same set of indicators formed the basis of a
\textbf{continuous risk index} that measured the degree of exposure to
exploitation. Each indicator was coded and weighted based on its
severity and frequency, and the weighted sum was normalised to create a
continuous measure ranging from low to high risk. This index allowed the
study to move beyond a simple dichotomy of exploited versus
non-exploited workers and instead conceptualise exploitation as a
spectrum of vulnerability. The weighting and normalisation procedures
were guided by expert consultation and reference to prior studies on
forced labour measurement.

Finally, the \textbf{knowledge and rights} module included questions on
awareness of employment rights, access to legal recourse, and
perceptions of fairness at work. While not directly used in the
prevalence estimation, these variables were important for contextual
interpretation of the results and for linking quantitative findings to
the qualitative insights reported in the companion paper by Yilmaz and
Emberson (2023).

Together, these modules provided the foundation for the dual-method
analysis. The inclusion of both ego and alter network questions was
intentional, enabling a direct comparison of RDS- and NSUM-based
prevalence estimates within a single, coherent survey design.

\subsection{Sample Characteristics}\label{sample-characteristics}

The final respondent-driven sampling (RDS) survey produced an analytic
sample of \emph{N = XX} domestic workers in the United Kingdom.
Participants were recruited through peer referral chains that extended
for up to \emph{X} waves from the initial seeds. Recruitment chains
differed by subgroup, reflecting variations in social network structure
and cohesion across the domestic work population.

The sample included three principal subgroups: Latinx (\emph{n = XX}),
Filipino (\emph{n = XX}), and British (\emph{n = XX}) domestic workers.
These subgroups were chosen because they represent the most prominent
national and linguistic groups within the UK domestic work sector.
Latinx respondents included workers primarily from Colombia, Bolivia,
and Ecuador; Filipino respondents were predominantly from Metro Manila
and surrounding provinces; and British respondents comprised individuals
engaged in domestic work either as part-time cleaners or caregivers.
Although these subgroups differ in migration histories and employment
arrangements, they share exposure to similar working environments and
structural vulnerabilities inherent in domestic employment.

Across the full sample, the majority of respondents were women (\emph{X
percent}), consistent with broader patterns in the feminisation of
domestic labour. The mean age of participants was \emph{XX years} (SD =
X). Over half reported working in live-in arrangements, while the
remainder were employed as live-out workers or part-time cleaners. The
distribution of employment types varied by subgroup: Latinx and Filipino
respondents were more likely to be employed as live-in domestic workers,
whereas British respondents were disproportionately represented among
part-time or hourly cleaning roles.

Network characteristics also differed across subgroups. Latinx
respondents tended to report larger average network sizes, with a mean
of \emph{X} known domestic workers, while British respondents reported
smaller and less interconnected networks. Filipino respondents occupied
an intermediate position, reflecting the long-established but socially
clustered Filipino domestic worker communities in several UK cities.
These differences are visible in the recruitment trees (Figure X), which
show longer, denser referral chains among Latinx respondents and
shorter, more fragmented structures among British participants. Such
variations are consistent with known differences in the density of
migrant community networks and have implications for both RDS weighting
and NSUM inference.

The geographic distribution of respondents was concentrated in London
and the South East of England, with smaller clusters in Manchester,
Birmingham, and Glasgow. This distribution reflects the spatial
concentration of private household employment in major metropolitan
areas. Many participants reported long working hours, limited rest
periods, and overlapping roles that combined cleaning, childcare, and
eldercare responsibilities. Among live-in workers, isolation from
external social networks was commonly reported.

Equilibrium diagnostics were conducted during recruitment to ensure
representativeness within the RDS framework. By the final wave,
equilibrium had been reached on key demographic and employment
variables, including subgroup identity, gender, and live-in status. The
convergence of sample proportions across waves indicated that the final
sample had stabilised with respect to these attributes, supporting the
validity of subsequent prevalence estimation.

Overall, the resulting sample provided a balanced representation of the
diversity within the UK domestic work sector. The composition of the
sample, combined with the use of network-based estimation methods,
allows for both quantitative assessment of exploitation prevalence and a
nuanced understanding of how risk varies across interconnected yet
distinct communities of domestic workers.

\subsection{Ethical Approval}\label{ethical-approval}

The data collection that underpins the analysis presented in this paper
was given favourable ethical approval by the lead author's School
Research Ethics Committee in January 2023. All participants were
informed about the aims of the study, provided informed consent, and
were assured that participation was voluntary and confidential.

\subsection{Use of Artificial Intelligence in
Research}\label{use-of-artificial-intelligence-in-research}

Large Language Models (LLMs) were used for brainstorming the
organisation of the paper and editing of text. Code co-pilot (Claude
Code) was used to test and debug R scripts employed in the statistical
analysis. No generative models were used to generate or simulate data.

\subsection{Estimation Framework}\label{estimation-framework}

\subsubsection{RDS Estimation}\label{rds-estimation}

RDS estimators use ego-based information to correct for unequal
inclusion probabilities. Each participant reports the size of their
network of other domestic workers, and estimators such as RDS-II and
Gile's successive sampling (SS) estimator (Gile 2011; Gile \& Handcock
2010) use this degree information and recruitment wave to weight
observations. For continuous traits, model-assisted inference approaches
(Gile 2011; Gile, Beaudry \& Handcock 2018) are applied to estimate
means and distributions of continuous outcomes.

\subsubsection{NSUM Estimation}\label{nsum-estimation}

NSUM relies on alter-based information, where respondents report how
many people they know who meet given criteria (e.g., ``How many domestic
workers do you know who have experienced exploitation?''). These alter
counts are used to infer prevalence across the population. Unlike RDS,
which corrects for recruitment bias, NSUM depends on the accuracy of
respondents' knowledge about their social contacts.

\subsubsection{Comparative Rationale}\label{comparative-rationale}

RDS and NSUM rely on distinct uses of network data: RDS focuses on ``who
recruited whom'' and degree-based weighting, while NSUM relies on ``how
many alters fit a criterion.'' Applying both methods to the same dataset
allows triangulation across these inferential logics.

\subsubsection{Bootstrap and
Sensitivity}\label{bootstrap-and-sensitivity}

To quantify uncertainty in NSUM estimates, we implemented a three-step
bootstrap procedure that resamples respondents, alters, and exploitation
classifications simultaneously, producing robust confidence intervals
(see Appendix C). Sensitivity analyses and model comparisons are
discussed in the Results section.


\printbibliography



\end{document}
